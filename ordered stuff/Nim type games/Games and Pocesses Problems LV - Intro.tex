%!TEX TS-program = XeLaTeX
%!TEX TS-program = XeLaTeX
\documentclass[11pt]{article}

\usepackage{amssymb}
\usepackage{amsthm}
\usepackage{amsmath}
\usepackage{mathtools}

\usepackage{fancyhdr}
\usepackage{graphicx}
\usepackage[top=3cm, left=2cm, right=2cm, headheight = 90pt]{geometry}
\usepackage{xltxtra}
\usepackage[font=small,labelfont=bf]{caption}

\usepackage{multicol}

\renewcommand{\theenumi}{\alph{enumi}}


\def\leq{\leqslant}
\def\geq{\geqslant}
\def\N{\mathbb N}
\def\R{\mathbb R}
\def\Z{\mathbb Z}
\DeclarePairedDelimiter\set\{\}

\def\prob{}

\theoremstyle{definition}
\newtheorem{problem}{\prob}


\pagestyle{fancy}

%!TEX TS-program = XeLaTeX

\fancyfoot[CE,CO]{}  % this is to remove page numbers (as you might want for single page docs)

%!TEX TS-program = XeLaTeX
\renewcommand{\figurename}{Attēls}

\fancyhead[C]{{\Large\bf Spēles - Ievaduzdevumi par Nim tipa spēlēm}\\ \date}

\renewcommand{\theenumi}{\alph{enumi}}


\begin{document}
%\thispagestyle{fancy}
\noindent 
%\emph{\notes}

%1 - Engel PSS Games, example
\begin{problem}
\textit{[Bacheta spēle]}

Uz galda ir $n$ akmentiņi. 
Spēlē divi spēlētāji A un B, A sāk. 
Legālā gājienā atļauts paņemt kādu no kopas $M$ skaitu ar akmentiņiem. 
Uzvar tas, kurš paņem pēdejo akmentiņu.
Atrodied zaudējo pozīciju kopu $L$, ja:
\begin{enumerate}
\item $M=\set{1,2,3,\dots ,k}$
\item $M=\set{1,2,4,\dots ,2^j, \dots}$ (visas nenegatīvās veselās divnieka pakāpes)
\item $M=\set{1,2,3,5,7,\dots }$ (1 un visi pirmskaitļi)
\item $M=\set{p^n}$, kur $p$ ir kāds pirmskaitlis, un $n$ ir patvaļīgs naturāls skaitlis %problem 2
\item $M=\set{1,3,8}$
\end{enumerate}

\end{problem}


%2 - Engel PSS Games, problem 5
\begin{problem}
\textit{[Zirdziņu šahs]}

A un B pamīšus liek katrs savas krāsas (baltus un melnus) zirdziņus uz šaha laukuma. Zirdziņu nevar novietot aizņemtā laukumā, vai laukumā, kuru apdraud pretējās krāsas zirdziņš. 

Vai zirdziņu šahā baltais var uzvarēt, pareizi spēlējot?

\end{problem}

%3 - ffa
\begin{problem}
\textit{[Šahs]}

Vai šaha spēlē baltie var uzvarēt vai vismaz garantēt neizšķirtu, pareizi spēlējot?

\end{problem}

%4 - Engel PSS Games, problem 18
\begin{problem}
\textit{[Dubultais šahs]}

Dubultajā šahā spēlētāji katrs pēc kārtas veic divus legālus gājienus.

Vai dubultā šaha spēlē baltie var uzvarēt vai vismaz garantēt neizšķirtu, pareizi spēlējot?

\end{problem}

%5 - Engel PSS Games, problem 21
\begin{problem}
\textit{[Nauda uz šaha]}

$n \times n$ šaha laukuma stūrī stāv 1 eiro monēta. Spēlētāji A un B pamīšus šo monētu pārvieto uz blakus stāvošu lauciņu ar ierobežojumu, ka monētu nevar likt uz lauciņa, kur tā jau kādreiz ir atradusies. Zaudē tas, kurš nevar izpildīt gājienu. 

Kurš uzvar pareizi spēlējot?

\end{problem}

%5 - Engel PSS Games, problem 1
\begin{problem}
\textit{[Wythoff'a spēle]}

Uz galda ir divas kaudzītes ar $x$ un $y$ akmentiņiem. Spēlētāji A un B pamīšus veic sekojošus gājienus: vai nu paņem jebkādu skaitu akmentiņu no vienas kaudzītes, vai nu vienādu skaitu akmentiņu no abām kaudzītēm. Uzvar tas, kurš paņem pēdējo akmentiņu. 

Kurš uzvar, pareizi spēlējot?

\end{problem}

%5 - Simplified IMO2012P3
\begin{problem}
\textit{Bonus problem - Vienkāršotā meļu spēle]}

Meļu minēšanas spēli spēlē divi spēlētāji A un B. 

Vispirms A izvēlas naturālu $ x \le 10$ patur to noslēpumā. 

Tālāk spēlētāja B mērķis ir iegūt informāciju par $x$, uzdodot A jautājumus sekojošā formā: norāda kādu skaitļu kopu $S$ (tā var neatķirties no iepriekš jautātas) un jautājuma, vai $x$ pieder šai kopai $S$. Spēlētājs A uzreiz atbild ar \textit{jā} vai \textit{nē}, taču viņš drīkst melot! Vienīgais ierobežojums - starp katrām $2$ secīgām atbildēm vismaz vienai atbildei jābūt patiesai.

Spēlētājs B var uzdot neierobežotu skaitu jautājumu, taču kādā brīdī viņam ir jānorāda kopa $X$, kas sastāv no ne vairāk kā $2$ skaitļiem. Ja  $x \in X$, B uzvar, citādi uzvar A. Pierādiet, ka B var garantēti uzvarēt!

\end{problem}

\end{document}

%!TEX TS-program = XeLaTeX
%!TEX TS-program = XeLaTeX
\documentclass[11pt]{article}

\usepackage{amssymb}
\usepackage{amsthm}
\usepackage{amsmath}
\usepackage{mathtools}

\usepackage{fancyhdr}
\usepackage{graphicx}
\usepackage[top=3cm, left=2cm, right=2cm, headheight = 90pt]{geometry}
\usepackage{xltxtra}
\usepackage[font=small,labelfont=bf]{caption}

\usepackage{multicol}

\renewcommand{\theenumi}{\alph{enumi}}


\def\leq{\leqslant}
\def\geq{\geqslant}
\def\N{\mathbb N}
\def\R{\mathbb R}
\def\Z{\mathbb Z}
\DeclarePairedDelimiter\set\{\}

\def\prob{}

\theoremstyle{definition}
\newtheorem{problem}{\prob}


\pagestyle{fancy}

%!TEX TS-program = XeLaTeX

\fancyfoot[CE,CO]{}  % this is to remove page numbers (as you might want for single page docs)

%!TEX TS-program = XeLaTeX
\renewcommand{\figurename}{Attēls}

\fancyhead[C]{{\Large\bf Skaitīšanas metodes - Uzdevumi}\\ \date}

\renewcommand{\theenumi}{\alph{enumi}}


\begin{document}
%\thispagestyle{fancy}
\noindent 
%\emph{\notes}

%1
\begin{problem}
\textit{[Ziemassvētku problēmas]}
Uz nākošajiem Ziemassvētkiem Valērija ir nolēmusi uzbūvēt $5$ dažādas kartona eglītes un tās izdekorēt.
\begin{enumerate}
\item Cik veidos Valērija var eglītes izkrāsot (katru eglīti vienā krāsā), ja viņai ir pieejamas trīs dažādu toņu krāsas?
\item Valērija ir iegādājusies $5$ dažādus eglīšu rotājumus. Cik veidos viņa var izdekorēt eglītes, ja viņa gribētu katru eglīti dekorēt ar tieši vienu rotājumu?
\item Cik daudz dažādu izgreznojuma kombināciju (krāsa + rotājums) ir iespējamas šīm $5$ eglītēm?
\item Cik dažādu veidu būs, ja Valērija pieļautu iespēju vairākus rotājumus karināt uz vienas eglītes? Un ja viņa pieļauj iespēju neizmantot visus rotājumus?
\end{enumerate}
\end{problem}
%

%2
\begin{problem}
\textit{[Neveselīga diēta]}
Skolas ēdnīcā ir konstanta $6$ ēdienu ēdienkarte. Romualds ir nolēmis baroties sekojoši – katru dienu ēst kaut kādus ēdienus (iespējams, neēst neko) tā, ka viņa izvēlēto ēdienu kopa atšķiras no viņa izvēlēm visās iepriekšējās dienās. 
Cik ir ilgākais periods, ko Romualds var izturēt, nepārkāpjot šos savus principus? Un kāds būs vidējais apēsto ēdienu skaits dienā?
\end{problem}
%

%3
\begin{problem}
\textit{[Veselīga diēta]}
Cilvēkēdāja pagrabā iespundēti $25$ gūstekņi. 
\begin{enumerate}
\item Cik veidos viņš var sastādīt sev dienas ēdienkarti rītdienai ($3$ ēdienreizes, katrā pa vienam gūsteknim un secība ir svarīga)?
\item Cik veidos viņš var atbrīvot tieši $3$ gūstekņus (secība nav svarīga)?
\end{enumerate}

\end{problem}
%

%4
\begin{problem}
\textit{[Deju problēma]}
Ja klubā ir $N$ puiši un $N$ meitenes, tad cik veidos viņi var sadalīties pa pāriem (lai dejotu)? Kā ir, ja puišu un meiteņu skaiti atšķiras? Un ja mēs atļaujam cilvēkiem negribēt dejot?

\end{problem}
%


%5
\begin{problem}
\textit{[Kvadrātu izkrāsošana]}
Cik veidos var izkrāsot sekojošas figūras, krāsojot katru rūtiņu vienā no divām krāsām?
\begin{enumerate}
\item  $2 \times 2$ rūtiņu kvadrātu
\item  $3 \times 3$ rūtiņu kvadrātu
\item  $2 \times 2$ rūtiņu kvadrātu,  ar papildus noteikumu, ka vienu krāsojumu nevar iegūt no kāda cita, rotējot šo kvadrātu?
\item  $3 \times 3$ rūtiņu kvadrātu,  ar papildus noteikumu, ka vienu krāsojumu nevar iegūt no kāda cita, rotējot šo kvadrātu?\footnote{Šis patiesībā ir grūts uzdevums}
\end{enumerate}
\end{problem}
%

%6
\begin{problem}
\textit{[Telefona numuri]}
Cik ir tādu dažādu astoņzīmju telefona numuru, kuri nesākas ar $0$?
\end{problem}
%

%7
\begin{problem}
\textit{[Ierobežoti sakārtojumi]}
Cik veidos var sakārtot skaitļus $\set{1..100}$ tā, lai blakus stāvoši skaitļi neatšķirtos vairāk kā par $1$?
\end{problem}
%

%8
\begin{problem}
\textit{[Paritāri skaitļi]}
Cik ir tādu astoņciparu skaitļu, kuru visiem cipariem\footnote{Ja nav minēts citādi, tad ir domāti cipari decimālajā pierakstā} ir vienāda paritāte?
\end{problem}
%

%9
\begin{problem}
\textit{[Nelīdzeni skaitļi]}
Cik ir tādu veselu skaitļu intervālā $(0..999999)$, kuru pierakstā nav sastopami divi vienādi blakusstāvoši cipari?
\end{problem}
%

%10
\begin{problem}
\textit{[Dilstoši skaitļi]}
Cik ir tādu sešciparu skaitļu, kuriem katrs nākošais cipars ir mazāks par iepriekšējo?
\end{problem}
%

%11
\begin{problem}
\textit{[Randiņa problēma]}
Ērika Džonatanam uz salvetes mēģināja uzrakstīt savu (astoņciparu) telefona numuru, bet alkohols darīja savu, un viņa izlaida vienu ciparu. Džonatans nākošā rītā izlēma šo problēmu risināt, pārlasot visus iespējamos telefona numurus. Cik numuru viņam ir jāpārlasa?
\end{problem}
%

%12
\begin{problem}
\textit{[Datumu problēma]}
ASV pieņemts datumu pierakstīt formātā MM-DD-YYYY. Eiropā, savukārt, DD-MM-YYYY. Cik ir tādu dienu gadā, kuru datumu nevar viennozīmīgi noteikt, nezinot izmantoto formātu?
\end{problem}

%13
\begin{problem}
\textit{[Kāpnīšu pastaiga]}
Par \textit{kāpnīšu pastaigu} sauksim tādu ceļu rūtiņu laipā starp punktiem $A$ un $B$, kura garums sakrīt ar īsāko šāda ceļa garumu. Cik dažādu kāpnīšu pastaigu eksistē starp:

\begin{enumerate}
\item Punktu $(0,0)$ un $(N,N)$?
\item Punktu $(0,0)$ un $(N,M)$?
\item Punktu $(0,0,0)$ un $(N,M,K)$, ja mēs šo vispārinam uz 3 dimensijām?
\end{enumerate}
\end{problem}

%14
\begin{problem}
\textit{[Kāpnīšu tikšanās]}
Kāda ir varbūtība satikties diviem ceļotājiem, kas ceļo diviem punktiem katrs savā virzienā un katrs izvēlas maršrutu pilnīgi patvaļīgi starp visām iespējamajām kāpnīšu pastaigām?
\begin{enumerate}
\item Punktu $(0,0)$ un $(N,N)$
\item Punktu $(0,0)$ un $(N,M)$
\item Punktu $(0,0,0)$ un $(N,M,K)$
\end{enumerate}
\end{problem}

%15
\begin{problem}
\textit{[Globāla problēma]}
Uz globusa ir novilktas $17$ paralēles un $24$ meridiāni. Cik daļās ir sadalīta globusa virsma?
\end{problem}
%

%16
\begin{problem}
\textit{[Piederības problēma]}
Ir dots $m \times n$ rūtiņu laukums. Tajā ir atzīmēta rūtiņa $(p,q)$. Cik ir tādu taisnstūru šajā rūtiņu laukumā (ar virsotnēm rūtiņu stūros), kuri satur šo rūtiņu?
\end{problem}
%

%17
\begin{problem}
\textit{[Kvadratenfrei*]}
\textit{Kvadrātbrīvs} skaitlis ir tāds, kurš nedalās ne ar vienu vesela skaitļa kvadrātu, izņemot {1}. Cik ir kvadrātbrīvu skaitļu intervālā $(1..100)$?
\end{problem}
%
\end{document}

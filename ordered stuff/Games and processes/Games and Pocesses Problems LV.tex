%!TEX TS-program = XeLaTeX
%!TEX TS-program = XeLaTeX
\documentclass[11pt]{article}

\usepackage{amssymb}
\usepackage{amsthm}
\usepackage{amsmath}
\usepackage{mathtools}

\usepackage{fancyhdr}
\usepackage{graphicx}
\usepackage[top=3cm, left=2cm, right=2cm, headheight = 90pt]{geometry}
\usepackage{xltxtra}
\usepackage[font=small,labelfont=bf]{caption}

\renewcommand{\theenumi}{\alph{enumi}}

\fancyfoot[CE,CO]{}  % this is to remove page numbers (as you might want for single page docs)

\def\leq{\leqslant}
\def\geq{\geqslant}
\def\N{\mathbb N}
\def\R{\mathbb R}
\def\Z{\mathbb Z}
\DeclarePairedDelimiter\set\{\}

\def\prob{}

\theoremstyle{definition}
\newtheorem{problem}{\prob}


\pagestyle{fancy}

%!TEX TS-program = XeLaTeX

\fancyfoot[CE,CO]{}  % this is to remove page numbers (as you might want for single page docs)

%!TEX TS-program = XeLaTeX
\renewcommand{\figurename}{Attēls}

\fancyhead[C]{{\Large\bf Spēles un Procesi - Uzdevumi}\\ \date}

\renewcommand{\theenumi}{\alph{enumi}}


\begin{document}
%\thispagestyle{fancy}
\noindent 
%\emph{\notes}

%1
\begin{problem}
$[IMO2009SLC1]$
Anna un Beāte spēlē sekojošu spēli. Uz gara galda rindā stāv 2009 kartiņas. Katrai kartiņai viena puse ir melna, bet otra - balta. Sākumā visas kartiņas ir pagrieztas ar melno pusi uz augšu. Katrā gajienā drīkst izvēlēties $50$ secīgas kartiņas tā, ka pati labējā ir melna un visas apgriezt uz pretējo pusi. Spēlētāja, kura nevar izdarīt gājienu, zaudē. Anna sāk. 

\begin{enumerate}
\item Vai šī spēle noteikti beidzas?
\item Vai Annai ir uzvaroša stratēģija šai spēlē?
\end{enumerate}

\end{problem}


%2
\begin{problem}
$[IMO2015SLC1]$
Virknē sarakstīti vairāki pozitīvi veseli skaitļi. Aija pati ar sevi spēlē sekojošu spēli - katrā gājienā viņa izvēlas divus blakusstāvošus skaitļus $x$ un $y$ tā, ka $x$ atrodas pa kreisi no $y$ un $x>y$, un aizstāj šo pāri ar vai nu pāri $(y+1,x)$ vai pāri $(x-1,x)$. \\
Pierādiet, ka šī rotaļa nevar turpināties bezgalīgi!\\
*Cik gājienus šī spēle var ilgt?
\end{problem}


%3
\begin{problem}
$[IMO2011SLC3]$
Kopa  $S$ satur vismaz divus plaknes punktus. Zināms, ka nekādi trīs no $S$ punktiem neatrodas uz vienas taisnes.\\
Par \textit{vējdzirnavām} sauksim sekojošu procesu. Sākam ar taisni $l$, kas iet caur punktu $P\in S$. Rotējam $l$ ap \textit{pagrieziena punktu} $P$ līdz taisne satur vēl kādu punktu $Q \in S$. Tagad $Q$ kļūst par pagrieziena punktu un vējdzirnavas turpina griezties.
Pierādiet, ka var izvelēties tādus $l$ un $P$, ka vējdzirnavas apmeklēs katru $S$ punktu bezgalīgi daudz reižu!
\end{problem}

%4
\begin{problem}
$[IMO2009SLC5]$
Pieci identiski 2-litru spaiņi izkārtoti pa apli. Pelnrušķīte un Pamāte spēlē sekojošu spēli. Pamāte savā gājienā no upes iesmeļ krūkā 1 litru ūdens un sadala šo ūdeni pa spaiņiem. Tad gājiens ir Pelnrušķītei un viņai atļauts no diviem blakusstāvošiem spaiņiem ūdeni izliet atpakaļ upē. Pamātes mērķis ir panākt, ka kāds spainis pārpildās. Pelnrušķītes mērķis ir panākt, lai neviens spainis nepārpildītos.
\begin{enumerate}
\item Kura uzvar šajā spēlē, pareizi spēlējot?
\item *Kādiem jābūt spaiņu tilpumiem, lai uzvarētu tā, kura neuzvar ar 2-litru spaiņiem?
\end{enumerate}
\end{problem}

%5
\begin{problem}
$[IMO2010SLC4]$
Rindā sakārtotas sešas monētu kaudzītes $S_1, S_2 ,.. S_6$. Sākumā katrā kaudzītē ir viena monēta. Ir pieļaujami divi gājienu veidi:
\renewcommand{\theenumi}{\roman{enumi}}
\begin{enumerate}
\item Ja kaudzītē $S_k, 1\leq k \leq 5$, ir vismaz viena monēta, to monētu var noņemt un kaudzītei $S_{k+1}$ pievienot divas monētas
\item Ja kaudzītē $S_k, 1\leq k \leq 4$, ir vismaz viena monēta, to monētu var noņemt un samainīt vietām kaudzītes $S_{k+1}$ un $S_{k+2}$
\end{enumerate}
Vai ir iespējams iegūt situāciju, kad pirmās piecas kaudzītes ir tukšas, bet sestā satur $2010^{2010}$ monētas? *Vai tas pats iespējams ar $2010^{2010^{2010}}$ monētām?
\end{problem}
\filbreak
%6
\begin{problem}
$[IMO2012P3]$
Meļu minēšanas spēli spēlē divi spēlētāji A un B. Spēles noteikumi ir parametrizēti ar naturāliem skaitļiem $k$ un $n$, kuri ir zināmi abiem spēlētājiem.

Vispirms A izvēlas naturālus $x$ un $N$ kuriem $1 \le x \le N$. Spēlētājs A skaitli $x$ patur noslēpumā, bet $N$ godīgi pasaka spēlētājam B. Tālāk spēlētāja B mērķis ir iegūt informāciju par $x$, uzdodot A jautājumus sekojošā formā: norāda kādu skaitļu kopu $S$ (tā var neatķirties no iepriekš jautātas) un jautājuma, vai $x$ pieder šai kopai $S$. Spēlētājs A uzreiz atbild ar \textit{jā} vai \textit{nē}, taču A var melot. Vienīgais ierobežojums - starp katrām $k+1$ secīgām atbildēm vismaz vienai atbildei jābūt patiesai.

Spēlētājs B var uzdot neierobežotu skaitu jautājumu, taču kādā brīdī viņam ir jānorāda kopa $X$, kas sastāv no ne vairāk kā $n$ skaitļiem. Ja  $x \in X$, B uzvar, citādi uzvar A. Pierādiet, ka:
\begin{enumerate}
\item Ja $n \ge 2^k$ tad B ir uzvaroša stratēģija.
\item Eksistē tāds naturāls $k_0$ ka katram $k \ge k_0$ eksistē tāds $n \ge 1.99^k$ kuram B nevar garantēti uzvarēt.
\end{enumerate}



\end{problem}

\end{document}

%!TEX TS-program = XeLaTeX
\documentclass[11pt]{article}

\usepackage{amssymb}
\usepackage{amsthm}
\usepackage{amsmath}
\usepackage{mathtools}

\usepackage{fancyhdr}
\usepackage{graphicx}
\usepackage[top=3cm, left=2cm, right=2cm, headheight = 90pt]{geometry}
\usepackage{xltxtra}
\usepackage[font=small,labelfont=bf]{caption}

%%%%%%%%%%%%%%    Language matters  %%%%

%\usepackage[latvian]{babel}
%\usepackage[L7x]{fontenc}
%\usepackage[utf8x]{inputenc}

%%%%%%%%%%%%%%%%%%%%%%%%%%%%%%%%%%%7%%%%%

%%%%%%%%%%%%%%%%%%%%%%%%%%%       DO NOT EDIT         %%%%%%%%%%%%%%%%%%
%\usepackage{setspace}
%\renewcommand{\headrulewidth}{1pt}
%\fancyhead[L]{\includegraphics[width=3cm]{pictures/logo}}
%\fancyhead[R]{\raisebox{3ex}{\fbox{Language: \bf \lang}}}
\fancyhead[C]{{\Large\bf Pigeonhole 1 solutions/comments}\\ \date}
\renewcommand{\theenumi}{\alph{enumi}}

\def\leq{\leqslant}
\def\geq{\geqslant}
\def\N{\mathbb N}
\def\R{\mathbb R}
\def\Z{\mathbb Z}

\DeclarePairedDelimiter\set\{\}
\newcommand\myeq{\stackrel{\mathclap{\normalfont\mbox{def}}}{=}}
\DeclarePairedDelimiter\ceil{\lceil}{\rceil}

%%%%%%%%%%%%%%%%%%%%%%%%%%%%%%%%%%%%%%%%%%%%%%%%%%%%%%%%%%%%%%%%%%%%%%%%%


%%% Language name in english %%%%%%%%%
\def\lang{Latvian}

%\def\lang{Lithuanian}

%%%%%%%%%%%%%%%%%%%%%%%%%%%%% TRANSLATE HERE %%%%%%%%%%%%%%%%%%%%%%%%%%%%%%%%%%

%\def\date{2018. gada 18. jūnijs}
%\def\notes{}


%%%%%%%%%%%%%%%%%%%%%%%%%%%%%%%%%%%%%%%%%%%%%%%%%%%%%%%%%%%%%%%%%%%%%%%%%%%%%%%

\def\prob{}

%%%%%%%%%%%%%%%%%%%%%%%%%%%%%%%%%%%%%%%%%%%%%%%%%%%%%%%%

\theoremstyle{definition}
\newtheorem{problem}{\prob}

\pagestyle{fancy}



\begin{document}
%\thispagestyle{fancy}
\noindent 
%\emph{\notes}

%1
\begin{problem}
\textit{[Hobbits]}

Well, this is the definition of the Pigeonhole Principle - if $n$ pigeons (or hobbits) all go into $m$ holes and $n>m$, then there exists a hole with at least $2$ pigeons in it.

\end{problem}

%1
\begin{problem}
\textit{[More hobbits]}

There should be a hole with at least $3$ hobbits, right?

\end{problem}

%1
\begin{problem}
\textit{[Even more hobbits]}

We might get to the \textit{Generalization 1}, that if $n=km+l$, then there exists a hole with at least $k+1$ pidgeon.

\end{problem}


\begin{problem}
\textit{[Football problem]}

In total $10m+1$ players have moved to Toland, so  if they are the pigeons and $m$ teams are the holes, then by Generalization 1 there exists a team with at least $11$ players in Toland.
\end{problem}

\begin{problem}
\textit{[Amount of friends]}

There can be $5$ possible number of friends someone can have in such group - from $0$ to $4$. However, if a group contains a person, who is friend with everyone else, then the group can 
\end{problem}

\begin{problem}
\textit{[Unique differences]}

There are $14$ possible differences between such numbers - from $1$ to $14$. These will be our holes. What should be the pidgeons? Well, the pairs of given numbers! However $8$ numbers make $28$ pairs\footnote{a good warmup/clue task for tactic of \textit{playing around and solving easier problems}}, which means that Pigeonhole does not give us what we need, immediately.

A little attention on the 'hole' $14$, however, helps - it becomes clear that $14$ as a difference between two distinct numbers that do not exceed $15$ can only be obtained from one pair of numbers - $1$ and $15$. So we say, that max $1$ pair goes into hole $14$, which leaves $27$ pairs and $13$ holes, and by Pigeonhole principle, we are done.
\end{problem}

\begin{problem}
\textit{[Too many kings]}
An important (potentially new) point -  In "\textbf{what is maximum/minimum}" type of problems, there are two parts - show that $n$ works and show that can be more/less than $n$.

Here we first show that there cant be more than $16$ kings on a board (not attacking each other), and then we show that $16$ is possible.

We divide the chess board into $16$ non-overlapping squares $2*2$. Obviously, each $2*2$ square can not contain more than $1$ king. 

Now, $2*2$ squares will be our holes, kings - pidgeons and by Pigeonhole follows that if we have more than $16$ kings on board, then at least one $2*2$ square will contain at least $2$ kings, which will be a problem.

Finally, to show that $16$ is achievable, we can put a king in upper-left corner of each $2*2$ square and it works. 
\end{problem}

\begin{problem}
\textit{[Too much money]}
This is very similar to previous problem, just with no grid to anchor on.
We divide $1m*1m$ into $25$ non-overlapping squares $20cm*20cm$. By Pigeonhole, one of these will contain at least 3 coins. We cover this square with our paper.

\end{problem}

\filbreak
\begin{problem}
\textit{[Existance of cliques]}

This one is a classic and we will come back to it in Graph Theory (also see \textit{Ramsay numbers}). 

Consider a person $A$. By Pigeonhole, he either has at least 3 acquaintances in the group, or at least 3 people he does not know - the $5$ remaining people are the pigdeons and two 'buckets' - 'people $A$ knows' and 'people $A$ does not know' - are the holes. 

Without loss of generality (maybe this is a place to talk about "\textbf{without loss of generality}" construction) assume that $A$ knows at least $3$ people - $B, C, D$. 
Now there are two possibilities - any of $B,C,D$ know each other or they don't. If former, and, say, $B$ and $D$ know each other, then we have found our triplet - $A, B, D$ who all know each other. If latter - none of $B, C, D$ know each other - we have found our triplet who do not know each other. 


\end{problem}
\filbreak

\begin{problem}
\textit{[Last digits of $3^x$]}

A nod to how Pigeonhole can appear outside combinatorics - for example, Number Theory. 

Tactical note - use \textit{Penultimate step} - number ends with $001$ if it is congruent to $1$ mod $1000$, i.e. $3^k -1$ divides by $1000$. If we multiply both sides by $3^l$ it still holds (since $3^x$ never divides by 1000). Therefore $3^l(x^k-1)= 3^{k+l}-3^l$. We do not proceed directly here, but rather oberve, that we have transformed what we need into a statement that difference of two powers of $3$ is divisible by $1000$ (tactical name for this - \textit{midpoint/step})

Now, difference of two numbers divides by $1000$ iff these two numbers have same remainder, when dividing by $1000$. Do such two powers of $3$ exist? Of course, by Pigeonhole! Holes are the possible remainders - $1000$ of them, but the pidgeons are powers of $3$ - an infinite number! (this is our second building block)

Putting it together, we say - let $3^n$ and $3^m$ be two powers such that they have same remainder, when dividing by 1000. Without loss of geerality assume $n>m$. Then $3^n-3^m$ divides by $1000$ evenly. But $3^n-3^m = 3^{m}(3^{n-m}-1)$. First part divides only by powers of 3, so $3^{n-m}-1$ must be dividing by $1000$ and therefore $3^{n-m}$ ends with $001$. 

\end{problem}
\filbreak

\begin{problem}
\textit{[College problems]}

This illustrates two things - \textit{hard work/perserverance} tactic and the idea that 'holes' can be quite abstract objects. 

We number courses $1..5$ and we describe each student as a subset of $\set{1,2,3,4,5}$. There are $32$ possible subsets\footnote{a warmup/intro/side problem- count the number of subsets}.
To use Pigeonhole principle (\textit{penultimate step} tactic) we would need to divide all possible subsets into $10$ groups so that if we take two subsets from one group, one will be subset of another.

Then we just use \textit{hard work} to construct these groups. For example, here within each group each subset is subset of all subsets to the right:
\renewcommand{\labelenumi}{\arabic{enumi}.}
\begin{enumerate}
\item
$[\emptyset, \set{1},\set{1,2}, \set{1,2,3},\set{1,2,3,4},\set{1,2,3,4,5}]$
\item
$[\set{2}, \set{2,5}, \set{1,2,5}, \set{1,2,3,5}]$
\item
$[\set{3}, \set{1,3}, \set{1,3,4}, \set{1,3,4,5}]$
\item
$[\set{4},  \set{1,4}, \set{1,2,4}, \set{1,2,4,5}]$
\item
$[\set{5}, \set{1,5}, \set{1,3,5}]$
\item
$[\set{2,4}, \set{2,4,5}, \set{2,3,4,5}]$
\item
$[\set{3,4}, \set{3,4,5}]$
\item
$[\set{3,5}, \set{2,3,5}]$
\item
$[\set{4,5}, \set{1,4,5}]$
\item
$[\set{2,3}, \set{2,3,4}]$
\end{enumerate}
Note - since each subset of size $2$ has to be in its own group, there can not be less than 10 such groups!
\end{problem}
\filbreak

\begin{problem}
\textit{Squares out of chaos]}

We start with some basic number theory - each number can be uniquely expressed as a multiplication of its prime factors raised to some power - in our case $a)$ we are limited to fist 9 prime factors: $n= p_1 ^{r_1} \cdot p_2 ^{r_2} \cdot ... \cdot p_{9} ^{r_{9}}$.

For the multiplication to be a perfect square, the sum of powers for all respective prime factors has to be an even number. I.e. if $a= p_1 ^{r_1^{(a)}} \cdot p_2 ^{r_2^{(a)}} \cdot ... \cdot p_{9} ^{r_{9}^{(a)}}$, $b= p_1 ^{r_1^{(b)}} \cdot p_2 ^{r_2^{(b)}} \cdot ... \cdot p_{9} ^{r_{9}^{(b)}}$ etc, then for $a \cdot b \cdot c \cdot d$ to be perfect square, all of sums $r_1^{(a)}+r_1^{(b)}+r_1^{(c)}+r_1^{(d)}$; $r_2^{(a)}+r_2^{(b)}+r_2^{(c)}+r_2^{(d)}$; ...; $r_{9}^{(a)}+r_{9}^{(b)}+r_{9}^{(c)}+r_{9}^{(d)} $ have to be even.
Since we are interested only in the parity of these sums, we are interested only in the parities of $r_i ^{(j)}$.

How many possible combinations of parities of powers of prime factors can a number have under condition $a)$? It's $2^{9}=512.$\footnote{warmup/side problem - count the number of binary numbers of lenght $n$} Since we have $2018$ numbers, by Pigeonhole principle, there will exist $4$ such numbers that their 'footprints' - the combination of parities of powers of prime factors - will be identical. These will be the numbers we were looking for in condition $a)$.

Now, in condition $b)$ we have $10$ possible prime factors, so the number of  possible 'footprints' is $2^{10}=1024$ and Pidgeonhole principle in its standard form does not help us. 

However we can make a following observation: either one of these 'footprints' have at least $4$ numbers (in which case we have the numbers we were looking for) or there exist at least $2$ 'footprints' with at least $2$ numbers in each (try to formalize this in Bonus problem).
In latter case, we take two pairs of numbers with pairwise identical 'footprints' and it is enough for sums of powers to be even, therefore giving us the numbers we were looking for. 
\end{problem}
\filbreak

\begin{problem}
\textit{[Bonus problem 1]}

This is an open-ended problem.
Some results could be:
\begin{itemize}

\item There are least $\ceil*{\frac{n}{c}}$ non empty cells
\item If $(n-o\cdot m) = k(c-o)+l$ for some $o,k \in \Z^*$, $l \in \N$, then exist at least $(k+1)$ holes with at least $o+1$ pidgeons
\end{itemize}

\end{problem}
\filbreak

\begin{problem}
\textit{[Bonus problem 2]}

Lets prove the basic form:\textit{If $n$ pigeons all go into $m$ holes and $n>m$, then there exists a hole with at least $2$ pigeons in it.}

\textbf{Proof by contradiction} (this could be new for someone): assume the opposite, that there is no hole with more than one pigeon. Then maximum amount of pigeons in the holes can be counted as $m*1=m$ pigeons - a contradiction with the fact that $n>m$. Therefore our assumption was incorrect and there is at least one hole with at least two pigeons.
\end{problem}
\filbreak

\end{document}

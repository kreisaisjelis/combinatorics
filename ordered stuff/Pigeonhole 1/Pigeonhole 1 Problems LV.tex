%!TEX TS-program = XeLaTeX
%!TEX TS-program = XeLaTeX
\documentclass[11pt]{article}

\usepackage{amssymb}
\usepackage{amsthm}
\usepackage{amsmath}
\usepackage{mathtools}

\usepackage{fancyhdr}
\usepackage{graphicx}
\usepackage[top=3cm, left=2cm, right=2cm, headheight = 90pt]{geometry}
\usepackage{xltxtra}
\usepackage[font=small,labelfont=bf]{caption}

\usepackage{multicol}

\renewcommand{\theenumi}{\alph{enumi}}


\def\leq{\leqslant}
\def\geq{\geqslant}
\def\N{\mathbb N}
\def\R{\mathbb R}
\def\Z{\mathbb Z}
\DeclarePairedDelimiter\set\{\}

\def\prob{}

\theoremstyle{definition}
\newtheorem{problem}{\prob}


\pagestyle{fancy}

%!TEX TS-program = XeLaTeX

\fancyfoot[CE,CO]{}  % this is to remove page numbers (as you might want for single page docs)

%!TEX TS-program = XeLaTeX
\renewcommand{\figurename}{Attēls}

\fancyhead[C]{{\Large\bf Dirihlē princips 1 - Uzdevumi}}

\begin{document}
%\thispagestyle{fancy}
\noindent 
%\emph{\notes}

%1
\begin{problem}
\textit{[Hobiti]}
$5$ hobiti rotaļājās meža pļaviņā. Paugurā pļavas malā bija $3$ hobitu alas. Pēkšņi no meža atskanēja orku kaujas saucieni un visi hobiti saskrēja alās. Ko ir iespējams pateikt par hobitu skaitu alās?
\end{problem}


%2
\begin{problem}
\textit{[Vairāk hobitu]}
Tas pats jautājums, bet ar $7$ hobitiem un $3$ hobitu alām?
\end{problem}

%3
\begin{problem}
\textit{[Vēl vairāk hobitu]}
Tas pats jautājums, bet ar $n$ hobitiem un $m$ hobitu alām?
\end{problem}

%4
\begin{problem}
\textit{[Futbola problēma]}
Kurmenistānā ir $m$ futbola komandas, katrā pa $11$ spēlētājiem. Visi šie spēlētāji ir savākušies lidostā, lai dotos uz futbola turnīru uz Turlandi. Lidmašīna veica $10$ reisus no Kurmenistānas un Turlandi un katru reizi aizveda $m$ spēlētājus. Vēl viens spēlētājs nenocietās un aizbrauca uz Turlandi ar autobusu. Pierādiet, ka Turlandē šobrīd vismaz viena no komandām ir pilnā sastāvā!

\end{problem}

%5
\begin{problem}
\textit{[Draugu daudzums]}
Pierādiet, ka jebkurā $5$ cilvēku kompānijā eksistē divi ar vienādu paziņu skaitu\footnote{šeit un turpmāk uzskatīsim, ka draudzības un pazīšanās ir abpusējas} šajā kompānijā!
\end{problem}

%6
\begin{problem}
\textit{[Unikālās atķirības]}
Doti $8$ dažādi naturāli skaitļi, kas nepārsniedz $15$. Pierādiet, ka starp šo skaitļu savstarpējām starpībām ir vismaz trīs vienādas!
\end{problem}

%7
\begin{problem}
\textit{[Pārāk daudz karaļu]}
Kādu maksimālo daudzumu karaļu var novietot uz šaha laukuma tā, lai neviens no tiem neapdraudētu citu?
\end{problem}

%8
\begin{problem}
\textit{[Pārāk daudz naudas]}
Kvadrātā $1\times 1$ metrs sameta $51$ (punktveida) monētu. Pierādiet, ka ar \mbox{$20\times 20$ cm} papīra lapu var pārklāt vismaz $3$ monētas!

\end{problem}

%9
\begin{problem}
\textit{[Kliķu neizbēgamība]}
Pierādiet, ka starp jebkuriem  $6$ cilvēkiem eksistē vai nu trīs savstarpēji pazīstami, vai nu trīs savstarpēji nepazīstami!
\end{problem}

%10
\begin{problem}
\textit{[$3^n$ pēdējie cipari]}
Pierādiet, ka eksistē skaitļa 3 pakāpe, kas beidzas ar cipariem $001$!
\end{problem}

%11
\begin{problem}
\textit{[Koledžas problēmas]}
$11$ studenti apmeklē $5$ lekcijas (ne obligāti visas). Pierādiet, ka eksistē tādi studenti $A$ un $B$, ka visas lekcijas, ko apmeklē $A$, apmeklē arī $B$!
\end{problem}

%12
\begin{problem}
\textit{[Kvadrāti no haosa]}
Kopa $A$ sastāv no $2020$ dažādiem naturāliem skaitļiem, visu šo skaitļu pirmreizinātāji ir mazāki par 
\renewcommand{\labelenumi}{\alph{enumi})}
\begin{enumerate}
\item
29
\item
30
\end{enumerate}
Pierādiet, ka kopā $A$ var atrast tādus $4$ dažādus skaitļus, ka $a\cdot b \cdot c \cdot d = n^2$ kādam naturālam $n$!
\end{problem}

%13
\begin{problem}
\textit{[Bonus uzdevums 1]}
Izdomājiet Dirihlē principa vispārinājumu – $a$ hobiti saskrēja $b$ alās, un zināms, ka alas maksimālā ietilpība ir $c$...
\end{problem}

%14
\begin{problem}
\textit{[Bonus uzdevums 2]}
Vai jūs varat pierādīt Dirihlē principu?
\end{problem}

\end{document}

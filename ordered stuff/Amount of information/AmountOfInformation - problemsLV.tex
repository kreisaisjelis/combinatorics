%!TEX TS-program = XeLaTeX
%!TEX TS-program = XeLaTeX
\documentclass[11pt]{article}

\usepackage{amssymb}
\usepackage{amsthm}
\usepackage{amsmath}
\usepackage{mathtools}

\usepackage{fancyhdr}
\usepackage{graphicx}
\usepackage[top=3cm, left=2cm, right=2cm, headheight = 90pt]{geometry}
\usepackage{xltxtra}
\usepackage[font=small,labelfont=bf]{caption}

\usepackage{multicol}

\renewcommand{\theenumi}{\alph{enumi}}


\def\leq{\leqslant}
\def\geq{\geqslant}
\def\N{\mathbb N}
\def\R{\mathbb R}
\def\Z{\mathbb Z}
\DeclarePairedDelimiter\set\{\}

\def\prob{}

\theoremstyle{definition}
\newtheorem{problem}{\prob}


\pagestyle{fancy}

%!TEX TS-program = XeLaTeX

\fancyfoot[CE,CO]{}  % this is to remove page numbers (as you might want for single page docs)

%!TEX TS-program = XeLaTeX
\renewcommand{\figurename}{Attēls}

\fancyhead[C]{{\Large\bf Informācijas daudzums - Uzdevumi}}

\begin{document}
%\thispagestyle{fancy}
\noindent 
%\emph{\notes}

%1
\begin{problem}
\textit{[Zeltkaļu ētika]}
Kādam citam Sultānam ļoti patīk bērni. Kādu reizi viņš noalgoja $100$ zeltkaļus uz gadu un uzdeva viņiem sekojošu uzdevumu - katru dienu gada garumā viņiem katram ir jāizgatavo rotaļlieta no precīzi $100$ gramiem tīra zelta un jāpiegāda to uz Sultāna pili. Tad, gada beigās, Sultāns plānoja apdāvināt visus sultanāta bērnus. 

Par nelaimi ir paklīdušas baumas, ka viens no zeltkaļiem ir negodīgs un šmaucās sekojošā veidā - katru dienu viņš piegādā rotaļlietu, kura sver $99$ gramus prasīto $100$ gramu vietā.

Sultānam ir pieejami digitāli svari, kuru precizitāte ir $1$ grams un maksimālais atļautais svars nav ierobežots. Sultāns ir arī diezgan slinks un tādēļ jautājums ir - ar kādu mazāko svēršanu daudzumu viņš var noskaidrot, vai un kurš no zeltkaļiem ir krāpnieks?
\end{problem}
%

%2
\begin{problem}
\textit{[Viltīgās monētas]}
Zeltkalei ir trīs identiska izskata monētas. Viņa zin, ka divas no tām ir īstas un sver vienādi, bet viena ir viltota un ir vai nu vieglāka, vai smagāka par īstajām. 
Viņai ir pieejami sviru svari - tiem ir divi kausi un katrā kausā kaut ko ieliekot svari parāda, kurš svaru kauss ir smagāks, vai arī, ka tie sver vienādi.

Kāds ir mazākais skaits svēršanu, kas viņai ir nepieciešamas, lai noteiktu, kura monēta ir viltota un vai tā ir smagāka, vai vieglāka? Cik svēršanas būtu vajadzīgas, ja viņa jau zinātu, ka viltotā monēta ir smagāka?

\end{problem}
%

%2
\begin{problem}
\textit{[Daudz viltīgu monētu - "smags darbs", "redukcija"]}
Tā pati zeltkale, tas pats uzdevums, bet tagad viņai ir $12$ monētas - $11$ īstas, un viena var būt viltota vai īsta. Viņa, grib noskaidrot, vai ir viltotā, un, ja ir, tad - kura, un - vai tā ir smagāka vai vieglāka. 
\end{problem}
%

\end{document}

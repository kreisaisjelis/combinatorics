%!TEX TS-program = XeLaTeX
%!TEX TS-program = XeLaTeX
\documentclass[11pt]{article}

\usepackage{amssymb}
\usepackage{amsthm}
\usepackage{amsmath}
\usepackage{mathtools}

\usepackage{fancyhdr}
\usepackage{graphicx}
\usepackage[top=3cm, left=2cm, right=2cm, headheight = 90pt]{geometry}
\usepackage{xltxtra}
\usepackage[font=small,labelfont=bf]{caption}

\usepackage{multicol}

\renewcommand{\theenumi}{\alph{enumi}}


\def\leq{\leqslant}
\def\geq{\geqslant}
\def\N{\mathbb N}
\def\R{\mathbb R}
\def\Z{\mathbb Z}
\DeclarePairedDelimiter\set\{\}

\def\prob{}

\theoremstyle{definition}
\newtheorem{problem}{\prob}


\pagestyle{fancy}

%%!TEX TS-program = XeLaTeX

\fancyfoot[CE,CO]{}  % this is to remove page numbers (as you might want for single page docs)

%%!TEX TS-program = XeLaTeX
\renewcommand{\figurename}{Attēls}

\fancyhead[C]{{\Large\bf Amount of information  - Solutions}}

\begin{document}
%\thispagestyle{fancy}
%\noindent 
%\emph{\notes}
\filbreak
%1
\begin{problem}
\textit{[Goldsmith ethics]}

\textbf{Problem}

Another Sultan was very fond of children. So much so, that he hired $100$ goldsmiths to work for him for a year and asked them each to make him one toy every day. He also specified that toy should be made of exactly $100$ grams of gold.

Unfortunately, someone started a rumour that there might be a goldsmith that is cheating in a following manner - all her toys weigh $99$ grams instead of required $100$. Sultan has a digital scale that shows him exact weigh of something with a precision of one gram. 

Sultan is also lazy, so - what is the minimum amount of weighings to check if this rumour is true and, if yes, find the offending goldsmith?

\textbf{Solution - Reduction to smaller, amount of information, symmetry}

Try to start with a simpler problem - say there is only one goldsmith. Then it is rather trivial - take any one of his toys, weigh it, if it weighs $99$ grams, then the goldsmith is a cheat.

Now - what if we have two goldsmiths - $A$ and $B$? Two weighings would be similarly trivial, but can we do it with one? Yes and this is the \textit{crux move} - the "heart" of the problem! But to show how to find it, let's introduce the argument through \textit{amount of information}.

If we are using the digital scale as described, the information we get from one weighing is the precise weight in grams - an integer number (in our case, since the scale has no maximum wight - the amount of information we can get in one weighing is unlimited). 
On other side - how many possible answers does our question have? Its $3$ - either $A$ is a cheat, or $B$ is a cheat or neither of them are cheats. This should tell us that it \textit{could} be possible with one weighing, but this also gives us a hint that in order to do this with one weighing, you have to construct this weighing in such a way that it is possible to get at least three different results! 

Add to that the argument that you have to include at least one toy from each goldsmith - otherwise you cant possibly hope to distinguish cases where the goldsmith we do not weigh toys from is a cheat or not. And add the symmetry argument that weighing can not have the same amount of toys from both goldsmiths - because, by symmetry, you wont be able to tell which of them is a cheat. 

What we are left with yields a simple solution - take one toy from $A$ and two toys from $B$. Possible outcomes - results of weighing - are:
\begin{itemize}
\item $300$ grams - means all three toys weigh $100$ grams and there are no cheats
\item $299$ grams - means that one toy was $99$ and since we know that if a goldsmith cheats, she cheats in all the toys, then the cheating goldsmith is $A$
\item $298$ grams - means that $B$ is a cheat with similar argument
\end{itemize}

Finally going back to $100$ goldsmiths is straightforward - take one toy from first, two from second, three from third etc. and $100$ toys from the last goldsmith (here we use the fact that a year has more than $100$ days). If the scale shows $X$ grams less than $1+2+\dots+99+100=101*50=5050$, then $X$ is also the number of goldsmith who is a cheat.

\end{problem}
%
\filbreak
%2
\begin{problem}
\textit{[Tricky coins]}

\textbf{Problem}

A goldsmith has three identical looking coins. He knows that two of them are real and identical in weight but one is fake and is either lighter or heavier than the real ones. She needs to find out which is the fake coin and if it is heavier or lighter.
She has balance scale - it is a scale that has two pans and shows you one of the three results - either contents of one of pans is heavier than other or that the contents are equal.

How many weighings does she need to do to determine the fake coin? Bonus question - what if she already knows that fake coin is heavier than real ones?


\textbf{Solution - Amount of information}

Again we look at \textit{amount of information} argument. How many possible outcomes can there be in one weighing on a balance scale? Its three - either left pan is heavier, lighter or equal to right pan.

On other side - how many possible answers are there to the question that she needs to answer? Its three possible coins that can be fake and then the fake can be lighter or heavier, so $3\times2=6$

From these two we can immediately conclude that one weighing is not going to be sufficient. This is the typical application of \textit{amount of information} method.

How many possible outcomes are there with two weighings? Well, since she can decide what she weighs in second weighing after the results on the first one, then it is $3\times3 = 9$ possible outcomes. So, two weighings seems possible - but this is not a proof that it is possible!

We show this upper bound by showing example of algorithm to determine the fake coin (and its lightness) with maximum $2$ weighings. 

How to construct such algorithm? Well, for first weighing of coins (call them $A$, $B$ and $C$) there  really aren't so may options, so, we set aside $C$ and weigh $A$ on one pan versus $B$ on the other (denote this by $A$ vs $B$). 
\begin{itemize}
\item $A>B$ ($A$ is heavier than $B$). Here we know that either $A$ is fake and is heavier than real ones or $B$ is fake and is lighter. We also know that $C$ is real because there is exactly one fake and it is $A$ or $B$. To determine which is which we can weigh, for example, $A$ vs $C$:
\begin{itemize}
\item $A>C$: this means that $A$ is fake and is heavier than others.   
\item $A=C$: this means that $B$ is fake and is lighter than others.   
\item $A<C$: this outcome is unreachable and will never happen.
\end{itemize}
\item $A>B$: this is an exercise in \textit{reduction} - it is exactly the same as previous item if you switch names $A$ and $B$.
\item $A=B$ ($A$ and $B$ weigh the same): here, obviously the fake coin is $C$, but we do not know, if it lighter or heavier. To figure that out, we use our second weighing, for example, $C$ vs $B$ (since we now know that $B$ must be real):
 \begin{itemize}
\item $C>B$: this means that $C$ is fake and is heavier than others.  
\item $C<B$: this means that $C$ is fake and is lighter than others.   
\item $C=B$: this is unreachable.
\end{itemize}
\end{itemize}

We are done, but to check our work it is suggested to check if all possible outcomes are reachable in this algorithm (and they seem to be).
\end{problem}
%
\filbreak
%2
\begin{problem}
\textit{[Many tricky coins - "hard work", "reduction", "penultimate step"]}

\textbf{Problem}

Same goldsmith, same question, but now she has $12$ coins - $11$ real and, possibly, but not necessarily, $1$ fake.
How many weighings does she need to do to determine the fake coin if it exists and whether it is heavier or lighter than real? 

\textbf{Solution}

First, the lower bound. In previous problem we already figured out (but did not explicitly state) that in $n$ weighings we can distinguish between maximum of $3^n$ outcomes. In this problem the question has $12\times 2 +1 = 25$ possible answers. Therefore we immediately can establish that two weighings will not be enough - because $3^2=9 <25$

To prove the upper bound - show how to do it with $3$ weighings - is simply to show the algorithm how to do it. But constructing this algorithm is not a trivial matter, however. Lets do this by using \textit{hard work} and \textit{useful notation} tactics! 

We can start with a reasoning like this - since our question has $25$ potential answers, and $3$ weighing algorithm gives maximum of $27$ answers, then we know that the algorithm we need to construct should be very \textit{efficient} - for example, all three results of the first weighing (<,>,=) should each still contain close to $9$ possibilities.  

Lets introduce some notations:
\begin{description}
\item[$n^!$] - will mean the number of weighings necessary to determine the fake coin if we know it exists between the $n$ coins
\item[$n^?$] - will mean the number of weighings necessary to determine the fake coin if the fake one may or may not exist between the $n$ coins
\end{description}

We can now consider, what can be the first weighing of a successful algorithm? More specifically, let's ask ourselves, how many coins can be not put on any of the pans in the first weighing? Say we have $k$ such coins. $k$ has to be even, because $12-k$ has split evenly into two pans.  
Now, if first weighing results in even pans, then it means that fake coin is between the $k$ coins left out of the first weighing. Since we still do not know whether the fake coin is lighter or heavier, then we have 
	reduced the $12^?$ to the $k^?$ with one weighing plus we know, from previous argument, that in this branch we still need to be able to discern at least $7$, but not more than $11$ results. From this directly follows, that $k$ has to be $4$ (by amount of information argument presented earlier).
	
Now we actually know what the first weighing of any successful algorithm for $12^?$ is - its $1,2,3,4$ vs $5,6,7,8$ and $9,10,11,12$ are left out of first weighing. Equality in this first weighing reduces the problem to $4^?$ and we will come back to that later, but now let's look at inequality, say $1,2,3,4<5,6,7,8$ Its not quite reduction to $8^!$ (which would be impossible with $2$ weighings!) because we know something more about these coins - we know that if $1,2,3$ or $4$ turns out to be fake, they will be lighter that real ones and, conversely, $5,6,7,8$ will be heavier - and the number of possible answers is actually only $8$.

Lets introduce some more notation!
\begin{description}
\item[$n_!$] - will mean the number of weighings necessary to determine the fake coin between the $n$ coins, if we know the \textit{forgery pattern} - i.e. if the first coin is fake, then it is heavier, if the second is fake, it is heavier, if the third is fake, it is lighter, $\dots$,  and if the last one is fake, then it is lighter, for example.\footnote{Since we are allowed to rearrange the coins freely, then what really matters in the forgery pattern is the number of coins that, if fake are lighter and heavier (and they are symmetrical)} 
\item[$n_?$] - will mean the number of weighings necessary to determine the fake coin between the $n$ coins, if we do not know the \textit{forgery pattern} - we do not know for all coins that if it is fake then it lighter (or heavier). 
\end{description}

So, we can use this new notation (or combination of both), that by using weighing $1,2,3,4$ vs $5,6,7,8$ we reduce $12^?_?$ to $4^?_?$ and $8^?_!$. What is left to show how to do $4^?_?$ and $8^?_!$ with two weighings that are left to us. Also note, that starting from the second weighing we have some extra resource - \textit{reference} coins that are known to be real. We might need them...

Lets start with $4^?_?$. Again, lets denote $m$ the amount of coins that are not weighed in the first of the two weighings of this sub-problem. Clearly, $m<2$, because $2^?_?$ would still have $5$ possible outcomes, so, by amount of information argument, it can not be solved by one remaining weighing.  On other hand, if $k=0$, that means that all $4$ coins took part in first weighing

\end{problem}
%

\end{document}

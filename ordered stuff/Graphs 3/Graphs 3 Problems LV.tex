%!TEX TS-program = XeLaTeX
%!TEX TS-program = XeLaTeX
\documentclass[11pt]{article}

\usepackage{amssymb}
\usepackage{amsthm}
\usepackage{amsmath}
\usepackage{mathtools}

\usepackage{fancyhdr}
\usepackage{graphicx}
\usepackage[top=3cm, left=2cm, right=2cm, headheight = 90pt]{geometry}
\usepackage{xltxtra}
\usepackage[font=small,labelfont=bf]{caption}

\usepackage{multicol}

\renewcommand{\theenumi}{\alph{enumi}}


\def\leq{\leqslant}
\def\geq{\geqslant}
\def\N{\mathbb N}
\def\R{\mathbb R}
\def\Z{\mathbb Z}
\DeclarePairedDelimiter\set\{\}

\def\prob{}

\theoremstyle{definition}
\newtheorem{problem}{\prob}


\pagestyle{fancy}

%!TEX TS-program = XeLaTeX

\fancyfoot[CE,CO]{}  % this is to remove page numbers (as you might want for single page docs)

%!TEX TS-program = XeLaTeX
\renewcommand{\figurename}{Attēls}

\fancyhead[C]{{\Large\bf Grafi 3 - Uzevumi}\\ \date}

\renewcommand{\theenumi}{\alph{enumi}}

\begin{document}

\noindent
 
\filbreak

%1
\begin{problem}
\textit{[BW1994PL19]}
Ziņkārzemes izlūkdienestam Tartu pilsētā ir $16$ spiegu. Katrs no tiem novēro kādus no saviem kolēģiem (vai nevienu). Zināms, ka, ja spiegs $A$ novēro spiegu $B$, tad spiegs $B$ nenovēro $A$. Tāpat ir zināms, ka katrus $10$ spiegus var sanumurēt tā, ka pirmais novēro otro, otrais - trešo, \dots, desmitais novēro pirmo.

Pierādiet, ka arī katrus $11$ spiegus var šādi sanumurēt!
\end{problem}
%

%2
\begin{problem}
\textit{[BW2010PL7]}
Ceļotājzemē ir vairākas pilsētas un viena no tām ir galvaspilsēta. Katrām divām pilsētām $A$ un $B$ ir tiešs avioreiss no $A$ uz $B$ un tiešs avioreiss no $B$ uz $A$ un zināms, ka abu virzienu cenas vienmēr ir vienādas. Papildus zināms, ka jebkurš maršruts, kas apceļo katru pilsētu vienu reizi un atgriežas izejas punktā, maksā vienādu summu.

Pierādiet, ka jebkurš ceļojums, kas apmeklē katru pilsētu, izņemot galvaspilsētu, tieši vienu reizi un atgriežas izejas punktā, arī maksā vienādu summu!
\end{problem}
%

%3
\begin{problem}
\textit{[IMO2007PL3]}
Matemātikas olimpiādē daži dalībnieki ir draugi. Draudzības ir abpusējas. Sauksim dalībnieku grupu par \textit{kliķi}, ja šajā grupā visi dalībnieki ir draugi (arī jebkura grupa no mazāk par diviem dalībniekiem ir kliķe). Ar \textit{kliķes izmēru} sapratīsim dalībnieku skaitu šajā kliķē.

Ja zināms, ka lielākās kliķes izmērs ir pāra skaitlis, pierādiet, ka visus olimpiādes dalībniekus var sadalīt divās telpās tā, ka lielākās kliķes izmērs katrā telpā ir vienāds!
\end{problem}
%

%4
\begin{problem}
\textit{[IMO2013SLC3]}

Fiziķis Kvantiņš nejauši atklāja jaunu elementārdaļiņu - \textit{imonu} un pēc neveiksmīga eksperimenta vairāki imoni parādījās viņa laboratorijā. Daži no imonu pāriem ir saistīti, un viens imons var izveidot vairākas saites. Kvantiņš ir atklājis, ka ar imoniem ir iespējams veikt divu veidu operācijas:
\begin{enumerate}
\item ja kāds imons ir saistīts ar nepāra skaitu imonu, tad šo imonu (un visas tā saites) var iznīcināt
\item ir iespējams nodublēt visu laboratorijas imonu kopu tā, ka katram imonam $I$ izveidojas atbilstošs imons $I^\prime$. Šī procesa laikā $I^\prime$ un $J^\prime$ sasaistās tad un tikai tad, ja to oriģināli $I$ un $J$ bija sasaistīti un, papildus tam, katrs $I$ sasaistās ar tā kopiju $I^\prime$. Citos brīžos saites neveidojas un nepazūd.
\end{enumerate}

Pierādiet, ka ar šīm operācijām Kvantiņš spēj iegūt pilnīgi nesaistītu imonu saimi!

\end{problem}
%

%4
\begin{problem}
\textit{[IMO2015SLC7]}
Cilvēku kompānijā daži cilvēki ir savstarpēji ienaidnieki. Cilvēku grupu sauksim par \textit{asociālu}, ja tās locekļu skaits ir nepāra un ir vismaz $3$, un šīs grupas cilvēkus var sasēdināt ap apaļu galdu tā, ka katri blakussēdošie ir ienaidnieki.

Ja zināms, ka kompānijā ir maksimums $2019$ asociālu grupu, pierādiet, ka ir iespējams šo kompāniju sadalīt $11$ telpās tā, ka nevieni divi ienaidnieki neatrodas vienā telpā!
\end{problem}
%
\end{document}

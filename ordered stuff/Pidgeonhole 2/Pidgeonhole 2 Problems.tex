%!TEX TS-program = XeLaTeX
%!TEX TS-program = XeLaTeX
\documentclass[11pt]{article}

\usepackage{amssymb}
\usepackage{amsthm}
\usepackage{amsmath}
\usepackage{mathtools}

\usepackage{fancyhdr}
\usepackage{graphicx}
\usepackage[top=3cm, left=2cm, right=2cm, headheight = 90pt]{geometry}
\usepackage{xltxtra}
\usepackage[font=small,labelfont=bf]{caption}

\renewcommand{\theenumi}{\alph{enumi}}

\fancyfoot[CE,CO]{}  % this is to remove page numbers (as you might want for single page docs)

\def\leq{\leqslant}
\def\geq{\geqslant}
\def\N{\mathbb N}
\def\R{\mathbb R}
\def\Z{\mathbb Z}
\DeclarePairedDelimiter\set\{\}

\def\prob{}

\theoremstyle{definition}
\newtheorem{problem}{\prob}


\pagestyle{fancy}

%!TEX TS-program = XeLaTeX

\fancyfoot[CE,CO]{}  % this is to remove page numbers (as you might want for single page docs)

%%!TEX TS-program = XeLaTeX
\renewcommand{\figurename}{Attēls}

\fancyhead[C]{{\Large\bf Dirihlē princips 2 - Uzdevumi}\\ \date}

\renewcommand{\theenumi}{\alph{enumi}}


\begin{document}
%\thispagestyle{fancy}
\noindent 
%\emph{\notes}

%1
\begin{problem}
\textit{[Kartupeļi]}

Kaujas lauks ir $10 \times 10$ rūtiņu laukums un Admirālim uz tā ir jāizvieto sekojoši kuģi: viens bāzes kuģis --- $1 \times 4$ rūtiņu taisntūris; divi līnijkuģi $1\times3$; trīs kreiseri $1\times2$; un četras fregates $1\times1$. Kuģiem jāatrodas uz laukuma rūtiņām un tie drīkst pieskarties laukuma malām, bet ne saskarties savā starpā. Pierādiet, ka

\begin{enumerate}
\item Ja Admirālis sāks ar lielāko kuģi un liks tos dilstošā secībā, tad viņai vienmēr izdosies izvietot visus kuģus
\item Ja Admirālis sāks ar kādu mazāku kuģi, ir iespējams, ka viņa nespēs novietot visus kuģus
\end{enumerate}
\end{problem}
%

%2
\begin{problem}
\textit{[Šķietamā izvēles brīvība]}
Laimai jāizvēlas $372$ dažādi skaitļi no kopas $\set{1,2,...,1200}$ tā, ka nekuru izvēlēto skaitļu starpība nav $4,5,$ vai $9$.

Pierādiet, ka Laima ir spiesta izvēlēties skaitli $600$!
\end{problem}
%


%3
\begin{problem}
\textit{[Dārgumu meklējumos]}
Miervaldis un Visvaldis spēlē sekojošu spēli. Eiklīda plaknē uzzīmēts liels riņķis. Miervaldis ir paslēpis dārgumus $n$ punktos riņķa iekšpusē. Visvaldis mēģina šos dārgumus atrast.

Visvaldis savā gājienā izvēlas punktu (riņķī vai ārpus tā) un Miervaldis paziņo attālumu no šī punkta līdz tuvākaja neatklātajam punktam, kas satur dārgumus. Ja šis attālums ir $0$, Miervaldis atklāj šo uzminēto dārgumu punktu. Visvaldim ir liels lineāls un ļoti liels cirkulis. 

Vai Visvaldis var atrast visus dārgumus ne vairāk kā $(n+1)^2$ gājienos?
\end{problem}
%

%4
\begin{problem}
\textit{[Kvadrāta šķelšana]}
Katra no $9$ taisnēm sadala kvadrātu divos četrstūros, kuru laukumiem ir proporcija $2:3$. Pierādiet, ka eksistē punkts, kurā satiekas vismaz trīs no šīm taisnēm!
\end{problem}
%

%5
\begin{problem}
\textit{[Kārtība no haosa]}
Skaitļi $\set{1,2,3,...,101}$ uzrakstīti rindā patvaļīgā secībā. Pierādiet, ka iespējams izdzēst $90$ no tiem tā, ka palikušie $11$ ir sakārtoti  augošā vai dilstošā secībā!
\end{problem}
%

%6
\begin{problem}
\textit{[Mindblowing trīsstūri]}
Bezgalīga laukuma rūtiņas ir izkrāsotas katra kādā no trim krāsām. Pierādiet, ka ir iespējams atrast taisnleņķa trīsstūri, kura visas virsotnes atrodas uz vienādas krāsas rūtiņām!
\end{problem}
%
\end{document}

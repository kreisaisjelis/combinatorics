%!TEX TS-program = XeLaTeX
%!TEX TS-program = XeLaTeX
\documentclass[11pt]{article}

\usepackage{amssymb}
\usepackage{amsthm}
\usepackage{amsmath}
\usepackage{mathtools}

\usepackage{fancyhdr}
\usepackage{graphicx}
\usepackage[top=3cm, left=2cm, right=2cm, headheight = 90pt]{geometry}
\usepackage{xltxtra}
\usepackage[font=small,labelfont=bf]{caption}

\usepackage{multicol}

\renewcommand{\theenumi}{\alph{enumi}}


\def\leq{\leqslant}
\def\geq{\geqslant}
\def\N{\mathbb N}
\def\R{\mathbb R}
\def\Z{\mathbb Z}
\DeclarePairedDelimiter\set\{\}

\def\prob{}

\theoremstyle{definition}
\newtheorem{problem}{\prob}


\pagestyle{fancy}

%!TEX TS-program = XeLaTeX

\fancyfoot[CE,CO]{}  % this is to remove page numbers (as you might want for single page docs)

%!TEX TS-program = XeLaTeX
\renewcommand{\figurename}{Attēls}

\fancyhead[C]{{\Large\bf Senioru mājas uzdevumi 2}}

\renewcommand{\theenumi}{\alph{enumi}}

\begin{document}

%\noindent
 
%\filbreak


\begin{problem}

Pierādiet, ka no $10$ patvaļīgi izvēlētiem dažādiem divciparu skaitļiem ir iespējams izvēlēties divas nešķeļošās apakškopas tā, ka to elementu summas ir vienādas.

\end{problem}

\begin{problem}

Dota kopa $S = \set{1, 2, 3, \dots ,1000000}$. Pierādiet, ka jebkurai $S$ apakškopai $A$, kas sastāv no $101$ elementa, ir iespējams atrast $100$ dažādus $S$ elementus $x_i$ tā, ka kopas $\set{a + x_i \mid a \in A}$ visas ir savstarpēji nešķeļošas.


\end{problem}


\begin{problem}

Par \emph{lokāciju} sauksim plaknes punktu $(x, y)$, ja $x$ un $y$ ir naturāli skaitļi, kas nepārsniedz $20$. 
Sākotnēji visas $400$ lokācijas ir brīvas.

Amēlija un Betānija spēlē spēli secīgi novietojot akmentiņus uz brīvām lokācijām. Sāk Amēlija un katrā savā gājienā novieto sarkanu akmentiņu uz brīvas lokācijas tā, ka starp sarkaniem akmentiņiem attālums nevar būt $\sqrt{5}$.
Betānija savā gājienā liek zilu akmentiņu uz brīvas lokācijas bez citiem ierobežojumiem.
Spēle beidzas, kad kāda no meitenēm nevar novietot akmentiņu. 

Atrodiet lielāko sarkano akmentiņu skaitu, ko Amēlija var garantēti novietot uz laukuma, neatkarīgi no tā, kā spēlē Betānija.


\end{problem}

\begin{problem}
Dots naturāls skaitlis $n$. 
Doti arī sviru svari un $n$ atsvariņi ar smagumiem $2^0,2^1, \cdots ,2^{n-1}$.
Mums šie atsvariņi visi jānovieto uz svariem, izvēloties tos kaut kādā secībā un liekot tos vienu pēc otra labajā vai kreisajā svaru kausā tā, ka nevienā brīdī labais kauss nav smagāks par kreiso.

Noskaidrojiet, cik veidos to ir iespējams izdarīt.
\end{problem}

\begin{problem}
Teiksim, ka plaknes punktu kopa $S$ ir \emph{sabalansēta}, ja katriem dažādiem punktiem $A,B \in S $, eksistē tāds $C \in S$, ka $AC=BC$.

Tiklab teiksim, ka kopa $S$ ir \emph{bezcentriska}, ja katriem trim $A,B,C \in S$, neeksistē tāds $P \in S$, ka $PA=PB=PC$.
\begin{enumerate}
\item Parādiet, ka visiem naturāliem $n\geq 3$, eksistē sabalansēta $n$ punktu kopa.
\item Atrodiet, kuriem naturāliem $n\geq 3$ eksistē sabalansēta bezcentriska $n$ punktu kopa.
\end{enumerate}

\end{problem}

%
\end{document}

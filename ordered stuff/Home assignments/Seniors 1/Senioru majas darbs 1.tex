%!TEX TS-program = XeLaTeX
%!TEX TS-program = XeLaTeX
\documentclass[11pt]{article}

\usepackage{amssymb}
\usepackage{amsthm}
\usepackage{amsmath}
\usepackage{mathtools}

\usepackage{fancyhdr}
\usepackage{graphicx}
\usepackage[top=3cm, left=2cm, right=2cm, headheight = 90pt]{geometry}
\usepackage{xltxtra}
\usepackage[font=small,labelfont=bf]{caption}

\renewcommand{\theenumi}{\alph{enumi}}

\fancyfoot[CE,CO]{}  % this is to remove page numbers (as you might want for single page docs)

\def\leq{\leqslant}
\def\geq{\geqslant}
\def\N{\mathbb N}
\def\R{\mathbb R}
\def\Z{\mathbb Z}
\DeclarePairedDelimiter\set\{\}

\def\prob{}

\theoremstyle{definition}
\newtheorem{problem}{\prob}


\pagestyle{fancy}

%!TEX TS-program = XeLaTeX

\fancyfoot[CE,CO]{}  % this is to remove page numbers (as you might want for single page docs)

%!TEX TS-program = XeLaTeX
\renewcommand{\figurename}{Attēls}

\fancyhead[C]{{\Large\bf Senioru mājas uzdevumi 1}\\ \date}

\renewcommand{\theenumi}{\alph{enumi}}

\begin{document}

\noindent
 
\filbreak

\begin{problem}
Alise un Beāte spēlē sekojošu spēli. Viņas ir uzrakstījuši katru no izteiksmēm $x+y$, $x-y$, $x^2+xy+y^2$ un $x^2-xy+y^2$ uz savas kārts. Šīs četras kārtis viņas sajauc un novieto uz galda ar izteiksmēm uz leju. Vienu no šīm kārtīm viņas apgriež, atklājot izteiksmi, pēc kā Alise izvēlas sev jebkuras divas no kārtīm un atlikušās divas dabū Beāte. Tad visas kārtis tiek atklātas. Tagad Alise izvēlas mainīgo $x$ vai $y$ un piešķir tam vērtību. Viņa parāda šo mainīgo un tā vērtību arī Beātei, kura tad piešķir vērtību otram mainīgajam (mainīgo vērtības var būt patvaļīgi reāli skaitļi).

Beigās katra meitene aprēķina savu divu izteiksmju reizinājumu un tad tos salīdzina. Kurai šis reizinājums ir lielāks, tā uzvar.
Kurai no spēlētājām ir uzvaroša stratēģija (ja kādai tāda vispār ir)?
\end{problem}

\begin{problem}
Atrodiet mazāko naturālo skaitli $k\geq 2$, kuram piemīt sekojoša īpašība:
jebkurā kopas $\{2,3,\ldots,k\}$ \mbox{sadalījumā} divās daļās vismaz viena no šīm daļām noteikti saturēs
(ne obligāti dažādus) skaitļus $a$, $b$ un~$c$,
kuriem $ab=c$.
\end{problem}

\begin{problem}
Baltijceļzemē ir $2019$ pilsētas. Dažas no tām ir savienotas ar divvirzienu ceļiem, kuri ārpus pilsētām krustojas ar viaduktiem. (Tas ir grafs, ne obligāti planārs.)  
Zināms, ka katram pilsētu pārim $A$ un $B$ ir iespējams nokļūt no $A$ uz $B$ braucot pa ne vairāk kā $2$ ceļiem.
Baltijceļzemē ir $62$ žandarmi un viens razbainieks, kuru žandarmi gribētu notvert.
Žandarmi un razbainieks jebkurā brīdī zin visu pārējo atrašanās vietu.
Katru nakti razbainieks var vai nu palikt tajā pilsētā, kurā viņš ir, vai arī pārvietoties uz blakus pilsētu (kas ar to ir savienota ar ceļu).
Katru dienu katram žandarmam ir šīs pašas iespējas un viņi var savas darbības saskaņot.
Ja kādā brīdi žandarms un razbainieks atrodas vienā pilsētā, tad razbainieks tiek noķerts.
Pierādiet, ka žandarmi, prātīgi rīkojoties, vienmēr varēs noķert razbainieku.
\end{problem}

\begin{problem}
Dotam naturālam $n$ aplūkosim visas neaugošās funkcijas ${f\colon\,\{1,\ldots,n\}\to\{1,\ldots,n\}}$.
Dažām no šīm funkcijām ir nekustīgais punkts (eksistē tāds $c$, kuram $f(c)=c$), bet citām --- nav. 
Nosakiet, par cik viena veida funkciju ir vairāk nekā otra.

\smallskip
\emph{Piebilde.} Funkcija \emph{$f$} ir \emph{neaugoša}, ja visiem $x \leq y$ izpildās $f(x)\geq f(y)$.
\end{problem}

\begin{problem}
Plaknē doti $2019$ punkti. Bērnelis vēlas uzzīmēt $k$ (slēgtus) riņķus tā, lai katriem diviem no šiem \mbox{punktiem} būtu riņķis, kurš saturētu tieši vienu no tiem.
Atrodiet mazāko $k$, pie kura to ir iespējams izdarīt jebkuram punktu izkārtojumam?
\end{problem}

%
\end{document}

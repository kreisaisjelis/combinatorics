%!TEX TS-program = XeLaTeX
%!TEX TS-program = XeLaTeX
\documentclass[11pt]{article}

\usepackage{amssymb}
\usepackage{amsthm}
\usepackage{amsmath}
\usepackage{mathtools}

\usepackage{fancyhdr}
\usepackage{graphicx}
\usepackage[top=3cm, left=2cm, right=2cm, headheight = 90pt]{geometry}
\usepackage{xltxtra}
\usepackage[font=small,labelfont=bf]{caption}

\renewcommand{\theenumi}{\alph{enumi}}

\fancyfoot[CE,CO]{}  % this is to remove page numbers (as you might want for single page docs)

\def\leq{\leqslant}
\def\geq{\geqslant}
\def\N{\mathbb N}
\def\R{\mathbb R}
\def\Z{\mathbb Z}
\DeclarePairedDelimiter\set\{\}

\def\prob{}

\theoremstyle{definition}
\newtheorem{problem}{\prob}


\pagestyle{fancy}

%!TEX TS-program = XeLaTeX

\fancyfoot[CE,CO]{}  % this is to remove page numbers (as you might want for single page docs)

%%!TEX TS-program = XeLaTeX
\renewcommand{\figurename}{Attēls}

\fancyhead[C]{{\Large\bf Senioru mājas darbi 5}}
%INDUCTION, a bit easyish



\renewcommand{\theenumi}{\alph{enumi}}

\begin{document}

\noindent
 
\filbreak


\begin{problem}
%https://www.problems.ru/view_problem_details_new.php?id=65210


Ančūrijā dienas vienmēr ir vai nu saulainas, vai lietainas. Ja kādu dienu laikapstākļi atšķiras no iepriekšējās dienas, ančūrieši saka, ka tajā dienā laiks ir mainījies. Ančūrijas zinātnieki ir noskaidrojuši, ka 1. janvārī vienmēr ir saulains, bet katrā nākamajā janvāra dienā saulains ir tad, ja pagājušā gada tajā pašā datumā laiks mainījās.

$2015.$ gada janvāris bija ļoti daudzveidīgs - laikapstākļi bieži mainījās. Kad ir nākošais gads, kurā laikapstākļi janvārī precīzi sakritīs ar $2015.$ gada janvāri?

\end{problem}

\begin{problem}
%https://artofproblemsolving.com/wiki/index.php/2008_USAMO_Problems/Problem_6


Kādā matemātikas konfrencē, katri divi matemātiķi ir vai nu draugi, vai ienaidnieki. Pusdienlaikā katrs dalībnieks ēd vienā no divām ēdamzālēm. Matemātiķi uzstāj, ka ēst var tikai zālē, kura satur pāra skaitu viņas draugu. Pierādiet, ka skaits, cik veidos var sadalīt matemātiķus starp ēdamzālēm atbilstoši šim nosacījumam ir $2^k$, kādam naturālam $k$.


\end{problem}



\begin{problem}
%https://artofproblemsolving.com/wiki/index.php/2005_Canadian_MO_Problems/Problem_3




Riņķa līnijas iekšienē atrodas $n>3$ punktu kopa $S$.
\begin {itemize}
\item Pierādiet, ka vienmēr var atrast trīs dažādus punktus $a,b,c \in S$ un trīs riņķa līnijas punktus $A,B,C$ tā, ka $a$ ir (stingri) tuvākais punktam $A$ no visas kopas $S$, $b$ ir vistuvākais $B$ un $c$ ir vistuvākais $C$. 
\item Pierādiet, ka nekādām $n$ vērtībām nevar garantēt $4$ šādu punktu pāru esamību! 
\end {itemize}
\end{problem}



%
\end{document}

%!TEX TS-program = XeLaTeX
%!TEX TS-program = XeLaTeX
\documentclass[11pt]{article}

\usepackage{amssymb}
\usepackage{amsthm}
\usepackage{amsmath}
\usepackage{mathtools}

\usepackage{fancyhdr}
\usepackage{graphicx}
\usepackage[top=3cm, left=2cm, right=2cm, headheight = 90pt]{geometry}
\usepackage{xltxtra}
\usepackage[font=small,labelfont=bf]{caption}

\usepackage{multicol}

\renewcommand{\theenumi}{\alph{enumi}}


\def\leq{\leqslant}
\def\geq{\geqslant}
\def\N{\mathbb N}
\def\R{\mathbb R}
\def\Z{\mathbb Z}
\DeclarePairedDelimiter\set\{\}

\def\prob{}

\theoremstyle{definition}
\newtheorem{problem}{\prob}


\pagestyle{fancy}

%!TEX TS-program = XeLaTeX

\fancyfoot[CE,CO]{}  % this is to remove page numbers (as you might want for single page docs)

%%!TEX TS-program = XeLaTeX
\renewcommand{\figurename}{Attēls}

\fancyhead[C]{{\Large\bf Homework  list 4}}
%INDUCTION, a bit easyish



\renewcommand{\theenumi}{\alph{enumi}}

\begin{document}

\noindent
 
\filbreak


\begin{problem}
%https://www.problems.ru/view_problem_details_new.php?id=65883

Uz horizontālas taisnes dažādos punktos ar veselām koordinātēm sēž galīgs skaits vardīšu. Katrā gājienā viena vardīte lec $1$ vienību uz labo pusi, pie kam tikai tad, ja piezemējoties tā neuzskrien citai vardītei. Mēs esam saskaitījuši, cik dažādos veidos vardītes var veikt $n$ gājienus (kādai fiksētai sākotnējai vardīšu konfigurācijai). Pierādiet, ka, ja nomainīt spēles noteikumus un ļaut vardītēm lēkt pa kreisi, nevis pa labi, tad $n$ gājienus varēs izdarīt tieši tikpat dažādos veidos!  


\end{problem}

\begin{problem}
%https://www.problems.ru/view_problem_details_new.php?id=66122
Čikāgā dzīvo $36$ gangsteri, daži no kuriem savā starpā naidojas. Katrs gangsteris sastāv vienā vai vairākās bandās, pie kam nav divu bandu, kurām precīzi sakristu to sastāvs. Ja gangsteri ir vienā bandā, tad tie savā starpā nenaidojas, taču, ja gangsteris nesastāv kādā bandā, tad viņa noteikti naidojas ar kādu šīs bandas dalībnieku.
Kāds ir lielākais skaits bandu, kas var būt Čikāgā? 


\end{problem}



\begin{problem}
%https://www.problems.ru/view_problem_details_new.php?id=105192
Apkārt riņķveida sienai ar garumu $1$ staigā divi punktveida sargi, pie kam viens no viņiem pārvietojas divreiz ātrāk nekā otrs. Šajā sienā ir logi. Logu izvietojumu sauc par \textit{drošu}, ja katrā brīdī vismaz viens sargs atrodas pie kāda loga, priekš kādas sākotnējās sargu atrašanās vietas. 
\begin {enumerate}
\item Ja logu sistēma sastāv tikai no viena loga un tā ir droša, kādam ir jābūt šī loga garumam?
\item Pierādiet, ka jebkuras drošas logu sistēmas summārais logu garums pārsniedz $\frac{1}{2}$
\item Pierādiet, ka jebkuram skaitlim $s>\frac{1}{2}$ eksistē droša logu sistēma ar summāro logu garumu, kas ir mazāks par $s$
\end {enumerate}

\end{problem}



%
\end{document}

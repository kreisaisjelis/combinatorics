%!TEX TS-program = XeLaTeX
%!TEX TS-program = XeLaTeX
\documentclass[11pt]{article}

\usepackage{amssymb}
\usepackage{amsthm}
\usepackage{amsmath}
\usepackage{mathtools}

\usepackage{fancyhdr}
\usepackage{graphicx}
\usepackage[top=3cm, left=2cm, right=2cm, headheight = 90pt]{geometry}
\usepackage{xltxtra}
\usepackage[font=small,labelfont=bf]{caption}

\renewcommand{\theenumi}{\alph{enumi}}

\fancyfoot[CE,CO]{}  % this is to remove page numbers (as you might want for single page docs)

\def\leq{\leqslant}
\def\geq{\geqslant}
\def\N{\mathbb N}
\def\R{\mathbb R}
\def\Z{\mathbb Z}
\DeclarePairedDelimiter\set\{\}

\def\prob{}

\theoremstyle{definition}
\newtheorem{problem}{\prob}


\pagestyle{fancy}

%!TEX TS-program = XeLaTeX

\fancyfoot[CE,CO]{}  % this is to remove page numbers (as you might want for single page docs)

%%!TEX TS-program = XeLaTeX
\renewcommand{\figurename}{Attēls}

\fancyhead[C]{{\Large\bf IMO SL  list 3}}
%INDUCTION, a bit easyish



\renewcommand{\theenumi}{\alph{enumi}}

\begin{document}

\noindent
 
\filbreak

\begin{problem}
%2013C1 -easy induction

Let $n$ be a positive integer. Find the smallest integer $k$ with the following property: Given
any real numbers $a_1, \dots , a_d$ such that $a_1 + a_2 + \dots + a_d = n$ and $0 \le a_i \le 1$ for $i = 1, 2, \dots , d$, it
is possible to partition these numbers into $k$ groups (some of which may be empty) such that the
sum of the numbers in each group is at most $1$.

\end{problem}

\begin{problem}
%2006C1 - med/easy induction?

We have $n \ge 2$ lamps $L_1, \dots, L_n$ in a row, each of them being either \textit{on} or \textit{off} . Every
second we simultaneously modify the state of each lamp as follows:
\begin{itemize}
\item if the lamp $L_i$ and its neighbours (only one neighbour for $i = 1$ or $i = n$, two neighbours for
other $i$) are in the same state, then $L_i$
is switched off;
\item otherwise, $L_i$ is switched on.
\end{itemize}

Initially all the lamps are off except the leftmost one which is on.
\begin{enumerate}
\item Prove that there are infinitely many integers $n$ for which all the lamps will eventually
be off.
\item Prove that there are infinitely many integers $n$ for which the lamps will never be all off.
\end{enumerate}


\end{problem}


\begin{problem}
%2010C2 - med/easy, induction
On some planet, there are $2^N$ countries $(N \ge 4)$. Each country has a flag $N$ units wide
and one unit high composed of $N$ fields of size $1\times1$, each field being either yellow or blue. No
two countries have the same flag.

We say that a set of $N$ flags is diverse if these flags can be arranged into an $N \times N$ square so
that all $N$ fields on its main diagonal will have the same color. Determine the smallest positive
integer $M$ such that among any $M$ distinct flags, there exist $N$ flags forming a diverse set.

\end{problem}




\begin{problem}
%2012C7 -hard graph induction

There are given $2^{500}$ points on a circle labeled $1, 2, \dots , 2^{500}$ in some order. Prove that
one can choose $100$ pairwise disjoint chords joining some of these points so that the $100$ sums
of the pairs of numbers at the endpoints of the chosen chords are equal.
\end{problem}





%
\end{document}

%!TEX TS-program = XeLaTeX
%!TEX TS-program = XeLaTeX
\documentclass[11pt]{article}

\usepackage{amssymb}
\usepackage{amsthm}
\usepackage{amsmath}
\usepackage{mathtools}

\usepackage{fancyhdr}
\usepackage{graphicx}
\usepackage[top=3cm, left=2cm, right=2cm, headheight = 90pt]{geometry}
\usepackage{xltxtra}
\usepackage[font=small,labelfont=bf]{caption}

\renewcommand{\theenumi}{\alph{enumi}}

\fancyfoot[CE,CO]{}  % this is to remove page numbers (as you might want for single page docs)

\def\leq{\leqslant}
\def\geq{\geqslant}
\def\N{\mathbb N}
\def\R{\mathbb R}
\def\Z{\mathbb Z}
\DeclarePairedDelimiter\set\{\}

\def\prob{}

\theoremstyle{definition}
\newtheorem{problem}{\prob}


\pagestyle{fancy}

%!TEX TS-program = XeLaTeX

\fancyfoot[CE,CO]{}  % this is to remove page numbers (as you might want for single page docs)

%!TEX TS-program = XeLaTeX
\renewcommand{\figurename}{Attēls}

\fancyhead[C]{{\Large\bf Senioru mājas uzdevumi 3}}
%INDUCTION, a bit easyish



\renewcommand{\theenumi}{\alph{enumi}}

\begin{document}

\noindent
 
\filbreak

\begin{problem}
%2013C1 -easy induction
Dots naturāls $n$. Atrodiet mazāko  $k$ ar sekojošu īpašību: ja doti reāli skaitļi $a_1, \dots , a_d$ kuriem $a_1 + a_2 + \dots + a_d = n$ un $0 \le a_i \le 1$, kur $i = 1, 2, \dots , d$, ir iespējams tos sadalīt $k$ grupās (dažas no tām var būt tukšas) tā, ka katras grupas skaitļu summa nepārsniedz $1$.

\end{problem}

\begin{problem}
%2006C1 - med/easy induction?

$n \ge 2$ lampas $L_1, \dots, L_n$ ir sakārtotas rindā, un katra no tām ir vai nu \textit{ieslēgta} vai \textit{izslēgta}. Katru sekundi visas lampas vienlaicīgi nomaina savu stāvokli sekojošā veidā:
\begin{itemize}
\item ja lampa $L_i$ un tās kaimiņi (pie $i = 1$ un $i = n$ lampai ir viens kaimiņš, citiem $i$ - divi kaimiņi) ir vienādā stāvoklī, tad $L_i$
izslēdzas;
\item citos gadījumos $L_i$ ieslēdzas.
\end{itemize}

Sākumā lampa $L_1$ ir ieslēgta, bet visas pārējās - izslēgtas.
\begin{enumerate}
\item Pierādiet, ka ir bezgalīgi daudz skaitļu $n$, kuriem kādā brīdī visas lampas ir izslēgtas.
\item Pierādiet, ka ir bezgalīgi daudz skaitļu $n$, kuriem nekad nepienāks brīdis, kad visas lampas ir izslēgtas.
\end{enumerate}


\end{problem}


\begin{problem}
%2010C2 - med/easy, induction
Uz kādas planētas ir $2^N$ valstis $(N \ge 4)$. Katrai valstij ir karogs $N$ vienību platumā un vienu vienību augstumā, kurš sastāv no $N$ rūtiņām $1\times1$, un katra rūtiņa ir vai nu zila vai dzeltena. Nevienām divām valstīm nav vienādu karogu.

Sauksim $N$ karogu kopu par \textit{diversificētu}, ja šos karogus var sakārtot $N \times N$ kvadrātā tā, ka visas $N$ rūtiņas uz kvadrāta galvenās diagonāles ir vienādā krāsā. Atrodiet mazāko naturālo $M$ tādu, ka starp jebkuriem dažādiem $M$ karogiem, var atrast $N$ karogus, kas veido diversificētu kopu.


\end{problem}




\begin{problem}
%2012C7 -hard graph induction
Uz riņķa līnijas atzīmēti $2^{500}$ punkti, kuriem piekārtoti skaitļi $1, 2, \dots , 2^{500}$ kaut kādā secībā. Pierādiet, ka iespējams izvēlēties $100$ savstarpēji nekrustojošas hordas, kas savieno kādus no šiem punktiem tā, ka $100$ summas no skaitļu pāriem, kuri piekārtoti hordas galapunktiem, ir vienādas.

\end{problem}





%
\end{document}

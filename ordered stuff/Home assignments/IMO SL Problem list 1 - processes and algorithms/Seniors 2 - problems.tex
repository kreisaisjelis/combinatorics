%!TEX TS-program = XeLaTeX
%!TEX TS-program = XeLaTeX
\documentclass[11pt]{article}

\usepackage{amssymb}
\usepackage{amsthm}
\usepackage{amsmath}
\usepackage{mathtools}

\usepackage{fancyhdr}
\usepackage{graphicx}
\usepackage[top=3cm, left=2cm, right=2cm, headheight = 90pt]{geometry}
\usepackage{xltxtra}
\usepackage[font=small,labelfont=bf]{caption}

\usepackage{multicol}

\renewcommand{\theenumi}{\alph{enumi}}


\def\leq{\leqslant}
\def\geq{\geqslant}
\def\N{\mathbb N}
\def\R{\mathbb R}
\def\Z{\mathbb Z}
\DeclarePairedDelimiter\set\{\}

\def\prob{}

\theoremstyle{definition}
\newtheorem{problem}{\prob}


\pagestyle{fancy}

%!TEX TS-program = XeLaTeX

\fancyfoot[CE,CO]{}  % this is to remove page numbers (as you might want for single page docs)

%!TEX TS-program = XeLaTeX
\renewcommand{\figurename}{Attēls}

\fancyhead[C]{{\Large\bf Senior home assignment 2}}

\renewcommand{\theenumi}{\alph{enumi}}

\begin{document}

\noindent
 
\filbreak

\begin{problem}
[IMO1972PL1SLC?]
Prove that from a set of ten distinct two-digit numbers (in the decimal system), it is possible to select two disjoint subsets whose members have the same sum.

\end{problem}

\begin{problem}
[IMO2003PL1SLC1]
$S$ is the set $\set{1, 2, 3, \dots ,1000000}$. Show that for any subset $A$ of $S$ with $101$ elements we can find $100$ distinct elements $x_i$ of $S$, such that the sets $\set{a + x_i \mid a \in A}$ are all pairwise disjoint.

\end{problem}


\begin{problem}
[IMO2018PL4SLC?]
A site is any point $(x, y)$ in the plane such that $x$ and $y$ are both positive integers less than or equal to 20. Initially, each of the 400 sites is unoccupied. Amy and Ben take turns placing stones with Amy going first. On her turn, Amy places a new red stone on an unoccupied site such that the distance between any two sites occupied by red stones is not equal to $\sqrt{5}$. On his turn, Ben places a new blue stone on any unoccupied site. (A site occupied by a blue stone is allowed to be at any distance from any other occupied site.) They stop as soon as a player cannot place a stone. Find the greatest $K$ such that Amy can ensure that she places at least $K$ red stones, no matter how Ben places his blue stones.

\end{problem}

\begin{problem}
[IMO2011PL4SLC?]
Let $n > 0$ be an integer. We are given a balance and $n$ weights of weight $2^0,2^1, \cdots ,2^{n-1}$. We are to place each of the $n$ weights on the balance, one after another, in such a way that the right pan is never heavier than the left pan. At each step we choose one of the weights that has not yet been placed on the balance, and place it on either the left pan or the right pan, until all of the weights have been placed. Determine the number of ways in which this can be done.

\end{problem}

\begin{problem}
[IMO2015PL1SLC2]
We say that a finite set $\mathcal{S}$ in the plane is balanced if, for any two different points $A$, $B$ in $\mathcal{S}$, there is a point $C$ in $\mathcal{S}$ such that $AC=BC$. We say that $\mathcal{S}$ is centre-free if for any three points $A$, $B$, $C$ in $\mathcal{S}$, there is no point $P$ in $\mathcal{S}$ such that $PA=PB=PC$.

Show that for all integers $n\geq 3$, there exists a balanced set consisting of $n$ points.
Determine all integers $n\geq 3$ for which there exists a balanced centre-free set consisting of $n$ points.

\end{problem}

%
\end{document}

%!TEX TS-program = XeLaTeX
%!TEX TS-program = XeLaTeX
\documentclass[11pt]{article}

\usepackage{amssymb}
\usepackage{amsthm}
\usepackage{amsmath}
\usepackage{mathtools}

\usepackage{fancyhdr}
\usepackage{graphicx}
\usepackage[top=3cm, left=2cm, right=2cm, headheight = 90pt]{geometry}
\usepackage{xltxtra}
\usepackage[font=small,labelfont=bf]{caption}

\usepackage{multicol}

\renewcommand{\theenumi}{\alph{enumi}}


\def\leq{\leqslant}
\def\geq{\geqslant}
\def\N{\mathbb N}
\def\R{\mathbb R}
\def\Z{\mathbb Z}
\DeclarePairedDelimiter\set\{\}

\def\prob{}

\theoremstyle{definition}
\newtheorem{problem}{\prob}


\pagestyle{fancy}

%!TEX TS-program = XeLaTeX

\fancyfoot[CE,CO]{}  % this is to remove page numbers (as you might want for single page docs)

%!TEX TS-program = XeLaTeX
\renewcommand{\figurename}{Attēls}

\fancyhead[C]{{\Large\bf IMO SL  list 1 - Processes and Algorithms - Problems}}

\renewcommand{\theenumi}{\alph{enumi}}

\begin{document}

\noindent
 
\filbreak

%2014C2 - easy, variant, AM-GM ineq, 
%2012C4 - coins in boxes, medium, long winded solution, forcing invariant
%2015C4 - number choosing game, medium, mirror-like strategy, not quite sure on one detail
%2010C6 - perl strings. medium-hard, good. Litle induction, insightful focus on values, case-work.
%2011C5 - ants on chessboard, nice, a bit geometrical, process manipulation for analyses. Medium hard.

\begin{problem}
%2014C2 - easy, variant, AM-GM ineq, 

We have $2^m$ sheets of paper, with the number $1$ written on each of them. We perform
the following operation: in every step we choose two distinct sheets; if the numbers on the two
sheets are $a$ and $b$, then we erase these numbers and write the number $a +  b$ on both sheets.

Prove that after $m2^{m-1}$ steps, the sum of the numbers on all the sheets is at least $4^m$.

\end{problem}

\begin{problem}
%2012C4 - coins in boxes, medium, long winded solution, forcing invariant

Players $A$ and $B$ play a game with $N \ge 2012$ coins and $2012$ boxes arranged around a circle. 
Initially $A$ distributes the coins among the boxes so that there is at least $1$ coin in each box. 
Then the two of them make moves in the order $B, A, B, A, \dots$ by the following rules:
\begin{itemize}
\item On every move of his $B$ passes $1$ coin from every box to an adjacent box.
\item On every move of hers $A$ chooses several coins that were not involved in $B$’s previous move and are in different boxes. She passes every chosen coin to an adjacent box.
\end{itemize}

Player $A$’s goal is to ensure at least $1$ coin in each box after every move of hers, regardless of how $B$ plays and how many moves are made. Find the least $N$ that enables her to succeed.


\end{problem}


\begin{problem}
%2015C4 - number choosing game, medium, mirror-like strategy, not quite sure on one detail

Let $n$ be a positive integer. Two players $A$ and $B$ play a game in which they take turns choosing positive integers $k \le n$. The rules of the game are:
\begin{itemize}
\item A player cannot choose a number that has been chosen by either player on any previous turn.
\item A player cannot choose a number consecutive to any of those \textit{this} player has chosen on any previous turn.
\item The game is a draw if all numbers have been chosen; otherwise the player who cannot choose a number anymore loses the game.
\end{itemize}
The player $A$ takes the first turn. Determine the outcome of the game, assuming that both players use optimal strategies.

\end{problem}

\begin{problem}
%2010C6 - perl strings. medium-hard, good. Litle induction, insightful focus on values, case-work.
Given a positive integer $k$ and other two integers $b > w > 1$. 
There are two strings of pearls, a string of $b$ black pearls and a string of $w$ white pearls. 
The length of a string is the number of pearls on it.
One cuts these strings in some steps by the following rules. In each step:
\begin {itemize}
\item The strings are ordered by their lengths in a non-increasing order. If there are some strings of equal lengths, then the white ones precede the black ones. Then $k$ first ones (if they
consist of more than one pearl) are chosen; if there are less than $k$ strings longer than $1$, then one chooses all of them.
\item Next, one cuts each chosen string into two parts differing in length by at most one.
\end{itemize}
(For instance, if there are strings of $5, 4, 4, 2$ black pearls, strings of $8, 4, 3$ white pearls and $k = 4$, then the strings of $8$ white, $5$ black, $4$ white and $4$ black pearls are cut into the parts $(4, 4), (3, 2), (2, 2)$ and $(2, 2)$, respectively.)
The process stops immediately after the step when a first isolated white pearl appears.
Prove that at this stage, there will still exist a string of at least two black pearls.

\end{problem}

\begin{problem}
%2011C5 - ants on chessboard, nice, a bit geometrical, process manipulation for analyses. Medium hard.
Let $m$ be a positive integer and consider a checkerboard consisting of $m$ by $m$ unit squares.
At the midpoints of some of these unit squares there is an ant. At time $0$, each ant starts
moving with speed $1$ parallel to some edge of the checkerboard. When two ants moving in
opposite directions meet, they both turn $90 \degree$ clockwise and continue moving with speed $1$.
When more than two ants meet, or when two ants moving in perpendicular directions meet, the ants continue moving in the same direction as before they met. 
When an ant reaches one of the edges of the checkerboard, it falls off and will not re-appear.
Considering all possible starting positions, determine the latest possible moment at which the last ant falls off the checkerboard or prove that such a moment does not necessarily exist.

\end{problem}

%
\end{document}

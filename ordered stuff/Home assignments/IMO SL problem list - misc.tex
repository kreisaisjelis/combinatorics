%!TEX TS-program = XeLaTeX
%!TEX TS-program = XeLaTeX
\documentclass[11pt]{article}

\usepackage{amssymb}
\usepackage{amsthm}
\usepackage{amsmath}
\usepackage{mathtools}

\usepackage{fancyhdr}
\usepackage{graphicx}
\usepackage[top=3cm, left=2cm, right=2cm, headheight = 90pt]{geometry}
\usepackage{xltxtra}
\usepackage[font=small,labelfont=bf]{caption}

\renewcommand{\theenumi}{\alph{enumi}}

\fancyfoot[CE,CO]{}  % this is to remove page numbers (as you might want for single page docs)

\def\leq{\leqslant}
\def\geq{\geqslant}
\def\N{\mathbb N}
\def\R{\mathbb R}
\def\Z{\mathbb Z}
\DeclarePairedDelimiter\set\{\}

\def\prob{}

\theoremstyle{definition}
\newtheorem{problem}{\prob}


\pagestyle{fancy}

%!TEX TS-program = XeLaTeX

\fancyfoot[CE,CO]{}  % this is to remove page numbers (as you might want for single page docs)

%!TEX TS-program = XeLaTeX
\renewcommand{\figurename}{Attēls}

\fancyhead[C]{{\Large\bf IMO SL list Misc - Problems}}

\renewcommand{\theenumi}{\alph{enumi}}

\begin{document}

\noindent
 
\filbreak

%2014C2 - easy, variant, AM-GM ineq, 
%2012C4 - coins in boxes, medium, long winded solution, forcing invariant
%2015C4 - number choosing game, medium, mirror-like strategy, not quite sure on one detail
%2010C6 - perl strings. medium-hard, good. Litle induction, insightful focus on values, case-work.
%2011C5 - ants on chessboard, nice, a bit geometrical, process manipulation for analyses. Medium hard.



\begin{problem}
%2010C6 - perl strings. medium-hard, good. Litle induction, insightful focus on values, case-work.
Given a positive integer $k$ and other two integers $b > w > 1$. 
There are two strings of pearls, a string of $b$ black pearls and a string of $w$ white pearls. 
The length of a string is the number of pearls on it.
One cuts these strings in some steps by the following rules. In each step:
\begin {itemize}
\item The strings are ordered by their lengths in a non-increasing order. If there are some strings of equal lengths, then the white ones precede the black ones. Then $k$ first ones (if they
consist of more than one pearl) are chosen; if there are less than $k$ strings longer than $1$, then one chooses all of them.
\item Next, one cuts each chosen string into two parts differing in length by at most one.
\end{itemize}
(For instance, if there are strings of $5, 4, 4, 2$ black pearls, strings of $8, 4, 3$ white pearls and $k = 4$, then the strings of $8$ white, $5$ black, $4$ white and $4$ black pearls are cut into the parts $(4, 4), (3, 2), (2, 2)$ and $(2, 2)$, respectively.)
The process stops immediately after the step when a first isolated white pearl appears.
Prove that at this stage, there will still exist a string of at least two black pearls.

\end{problem}

\begin{problem}
%2011C7 - ok, hardish
On a square table of $2011$ by $2011$ cells we place a finite number of napkins that each cover a square of $52$ by $52$ cells. In each cell we write the number of napkins covering it, and we record the maximal number $k$ of cells that all contain the same nonzero number. Considering all possible napkin configurations, what is the largest value of $k$?

\end{problem}


%
\end{document}

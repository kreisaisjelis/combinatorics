%!TEX TS-program = XeLaTeX
\documentclass[11pt]{article}

\usepackage{amssymb}
\usepackage{amsthm}
\usepackage{amsmath}

\usepackage{fancyhdr}
\usepackage{graphicx}
\usepackage[top=3cm, left=2cm, right=2cm, headheight = 90pt]{geometry}
\usepackage{xltxtra}

%%%%%%%%%%%%%%    Language matters  %%%%

%\usepackage[latvian]{babel}
%\usepackage[L7x]{fontenc}
%\usepackage[utf8x]{inputenc}

%%%%%%%%%%%%%%%%%%%%%%%%%%%%%%%%%%%7%%%%%

%%%%%%%%%%%%%%%%%%%%%%%%%%%       DO NOT EDIT         %%%%%%%%%%%%%%%%%%
%\usepackage{setspace}
%\renewcommand{\headrulewidth}{1pt}
%\fancyhead[L]{\includegraphics[width=3cm]{pictures/logo}}
%\fancyhead[R]{\raisebox{3ex}{\fbox{Language: \bf \lang}}}
\fancyhead[C]{{\Large\bf Kombinatoriskās ģeometrijas uzdevumi}\\ \date}
\renewcommand{\theenumi}{\alph{enumi}}

\def\leq{\leqslant}
\def\geq{\geqslant}
\def\N{\mathbb N}
\def\R{\mathbb R}
\def\Z{\mathbb Z}

%%%%%%%%%%%%%%%%%%%%%%%%%%%%%%%%%%%%%%%%%%%%%%%%%%%%%%%%%%%%%%%%%%%%%%%%%


%%% Language name in english %%%%%%%%%
\def\lang{Latvian}

%\def\lang{Lithuanian}

%%%%%%%%%%%%%%%%%%%%%%%%%%%%% TRANSLATE HERE %%%%%%%%%%%%%%%%%%%%%%%%%%%%%%%%%%

\def\date{2018. gada 18. jūnijs}
%\def\notes{}


%%%%%%%%%%%%%%%%%%%%%%%%%%%%%%%%%%%%%%%%%%%%%%%%%%%%%%%%%%%%%%%%%%%%%%%%%%%%%%%

\def\prob{}

%%%%%%%%%%%%%%%%%%%%%%%%%%%%%%%%%%%%%%%%%%%%%%%%%%%%%%%%

\theoremstyle{definition}
\newtheorem{problem}{\prob}

\pagestyle{fancy}
\fancyfoot[CE,CO]{}  % this is to remove page numbers (as you might want for single page docs)


\begin{document}
%\thispagestyle{fancy}
\noindent 
%\emph{\notes}

%1
\begin{problem}
$[BW1999]$
Vai diska ar rādiusu $1$ (ieskaitot aploci) punktus var sadalīt trīs apakškopās tā, lai nevienā apakškopā neatrastos divi punkti, kuru attālums ir tieši $1$?
\end{problem}

%3
\begin{problem}
$[IMO2013PL2]$
$n$ sarkanu un $n+1$ zilu punktu izvietojums plaknē saucas par Kolumbisku, ja nevieni trīs punkti neatrodas uz vienas taisnes. Vairāku taišņu izvietojumu šajā plaknē sauc par pūderainu, ja neviena taisne nesatur nevienu no Kolumbiskā izkārtojuma punktiem un neviens no reģioniem, kuros plakni sadala šīs taisnes, nesatur abu krāsu punktus. 

Atrodiet mazāko $k$, priekš kura katram Kolumbiskam $4027$ punktu izkārtojumam eksistē pūderains $k$ taišņu izvietojums!
\end{problem}

%5
\begin{problem}
$[IMO2017PL3]$
Medniece un neredzams trusītis spēlē spēli Eiklīda plaknē. Trusīša sākuma atrašanās punkts $A_0$ sakrīt ar mednieces sākuma atrašanās punktu $B_0$. Pēc $n-1$ gājiena, trusītis atrodas punktā $A_{n-1}$ un medniece atrodas punktā $B_{n-1}$. Gājienā numur $n$ secīgi notiek sekojošas lietas.
\begin{itemize}
\item[(i)]
Trusītis, medniecei neredzot, pārvietojas uz punktu $A_n$ tā, ka attālums starp $A_{n-1}$ un $A_n$ ir tieši 1.
\item[(ii)]
Radars paziņo medniecei punktu $P_n$. Radars garantē tikai to, ka attālums starp $P_n$ un $A_n$ nepārsniedz 1.
\item[(iii)]
Medniece, trusītim redzot, parvietojas uz punktu $B_n$ tā, ka attālums starp $B_{n-1}$ un $B_n$ ir \newline tieši 1.
\end{itemize}

Vai, lai kā nekustētos trusītis un kādus punktus nerādītu radars, medniece var izvēlēties savus gājienus tā, ka pēc $10^9$ gājieniem viņa var garantēt, ka attālums starp viņu un trusīti nepārsniedz $100$?
\end{problem}
\end{document}

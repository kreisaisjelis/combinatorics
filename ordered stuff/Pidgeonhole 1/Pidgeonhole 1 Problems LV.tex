%!TEX TS-program = XeLaTeX
%!TEX TS-program = XeLaTeX
\documentclass[11pt]{article}

\usepackage{amssymb}
\usepackage{amsthm}
\usepackage{amsmath}
\usepackage{mathtools}

\usepackage{fancyhdr}
\usepackage{graphicx}
\usepackage[top=3cm, left=2cm, right=2cm, headheight = 90pt]{geometry}
\usepackage{xltxtra}
\usepackage[font=small,labelfont=bf]{caption}

\renewcommand{\theenumi}{\alph{enumi}}

\fancyfoot[CE,CO]{}  % this is to remove page numbers (as you might want for single page docs)

\def\leq{\leqslant}
\def\geq{\geqslant}
\def\N{\mathbb N}
\def\R{\mathbb R}
\def\Z{\mathbb Z}
\DeclarePairedDelimiter\set\{\}

\def\prob{}

\theoremstyle{definition}
\newtheorem{problem}{\prob}


\pagestyle{fancy}

%!TEX TS-program = XeLaTeX

\fancyfoot[CE,CO]{}  % this is to remove page numbers (as you might want for single page docs)

%!TEX TS-program = XeLaTeX
\renewcommand{\figurename}{Attēls}

\fancyhead[C]{{\Large\bf Dirihlē 1 - Uzdevumi}\\ \date}

\begin{document}
%\thispagestyle{fancy}
\noindent 
%\emph{\notes}

%1
\begin{problem}
\textit{[Hobbits]}
$5$ hobbits were frolicking on a meadow. There were $3$ hobbit holes nearby. Suddenly an orcish war cry sounded from the forest and hobbits all ran and hid in hobbit holes. What can you say about number of hobbits in hobbit holes?
\end{problem}


%2
\begin{problem}
\textit{[More hobbits]}
What if there are $7$ hobbits and $3$ hobbit holes?
\end{problem}

%3
\begin{problem}
\textit{[Even more hobbits]}
What if there are $n$ hobbits and $m$ hobbit holes?
\end{problem}

%4
\begin{problem}
\textit{[Football problem]}
There are $m$ football teams in Fromenistan, each has $11$ players. All these players have gathered in an airport to go to tournament in Toland. Airplane has already made 10 flights each time bringing $m$ footballers to Toland. One more footballer was so impatient that he went to Toland by bus (and has already arrived).  

Prove that there is at least one full team in Toland already!
\end{problem}

%5
\begin{problem}
\textit{[Amount of friends]}
Prove that any company of $5$ people has at least two persons that have the same amount of friends\footnote{herefrom we always consider friendships to be bi-directional} within this group!
\end{problem}

%6
\begin{problem}
\textit{[Unique differences]}
We have $8$ distinct natural numbers that do not exceed $15$. Prove that there are at least three equal differences between these numbers!
\end{problem}

%7
\begin{problem}
\textit{[Too many kings]}
What is the maximum number of kings that can be placed on a chess board so that none of them can attack another \footnote{standard chess board is $8*8$ squares, king attacks all nearby squares (including diagonals)}?
\end{problem}

%8
\begin{problem}
\textit{[Too much money]}
We randomly dropped $51$ pointlike coins into a square with side length of $1$ meter. Prove that it is always possible to cover at least three coins with a square paper sheet $20cm * 20cm$ !
\end{problem}

%9
\begin{problem}
\textit{[Existance of cliques]}
Prove that between any $6$ people it is always possible to find either three that know each other, or three that do not know each other!
\end{problem}

%10
\begin{problem}
\textit{[Last digits of $3^x$]}
Prove that there exists such a natural $n$ that $3^n$ ends with digits\footnote{always assume decimal, unless stated otherwise} $001$ !
\end{problem}

%11
\begin{problem}
\textit{[College problems]}
In the college there are $11$ students and $5$ courses. Prove that such two students $A$ and $B$ exist that all courses attended by $A$ is also attended by $B$!  
\end{problem}

%12
\begin{problem}
\textit{[Squares out of chaos]}
Set $A$ consists of $2018$ distinct numbers. All prime factors of all these numbers are
\renewcommand{\labelenumi}{\alph{enumi})}
\begin{enumerate}
\item
less than 29
\item
less than 30
\end{enumerate}
Prove that $A$ contains $4$ distinct numbers $a,b,c,d$, such that $a\cdot b \cdot c \cdot d = n^2$ for some natural number $n$!
\end{problem}

%13
\begin{problem}
\textit{[Bonus problem 1]}
Come up with a generalization of Pidgeonhole principle - What can we know, if $a$ hobbits all went into $b$ hobbit-holes and it is known that maximum capacity of a hobbit hole is $c$?
\end{problem}

%14
\begin{problem}
\textit{[Bonus problem 2]}
Can you prove Pidgeonhole principle?
\end{problem}

\end{document}
=======
%!TEX TS-program = XeLaTeX
%!TEX TS-program = XeLaTeX
\documentclass[11pt]{article}

\usepackage{amssymb}
\usepackage{amsthm}
\usepackage{amsmath}
\usepackage{mathtools}

\usepackage{fancyhdr}
\usepackage{graphicx}
\usepackage[top=3cm, left=2cm, right=2cm, headheight = 90pt]{geometry}
\usepackage{xltxtra}
\usepackage[font=small,labelfont=bf]{caption}

\renewcommand{\theenumi}{\alph{enumi}}

\fancyfoot[CE,CO]{}  % this is to remove page numbers (as you might want for single page docs)

\def\leq{\leqslant}
\def\geq{\geqslant}
\def\N{\mathbb N}
\def\R{\mathbb R}
\def\Z{\mathbb Z}
\DeclarePairedDelimiter\set\{\}

\def\prob{}

\theoremstyle{definition}
\newtheorem{problem}{\prob}


\pagestyle{fancy}

%!TEX TS-program = XeLaTeX

\fancyfoot[CE,CO]{}  % this is to remove page numbers (as you might want for single page docs)

%!TEX TS-program = XeLaTeX
\renewcommand{\figurename}{Attēls}

\fancyhead[C]{{\Large\bf Dirihlē princips 1 - Uzdevumi}}

\begin{document}
%\thispagestyle{fancy}
\noindent 
%\emph{\notes}

%1
\begin{problem}
\textit{[Hobiti]}
$5$ hobiti rotaļājās meža pļaviņā. Paugurā pļavas malā bija $3$ hobitu alas. Pēkšņi no meža atskanēja orku kaujas saucieni un visi hobiti saskrēja alās. Ko ir iespējams pateikt par hobitu skaitu alās?
\end{problem}


%2
\begin{problem}
\textit{[Vairāk hobitu]}
Tas pats jautājums, bet ar $7$ hobitiem un $3$ hobitu alām?
\end{problem}

%3
\begin{problem}
\textit{[Vēl vairāk hobitu]}
Tas pats jautājums, bet ar $n$ hobitiem un $m$ hobitu alām?
\end{problem}

%4
\begin{problem}
\textit{[Futbola problēma]}
Kurmenistānā ir $m$ futbola komandas, katrā pa $11$ spēlētājiem. Visi šie spēlētāji ir savākušies lidostā, lai dotos uz futbola turnīru uz Turlandi. Lidmašīna veica $10$ reisus no Kurmenistānas un Turlandi un katru reizi aizveda $m$ spēlētājus. Vēl viens spēlētājs nenocietās un aizbrauca uz Turlandi ar autobusu. Pierādiet, ka Turlandē šobrīd vismaz viena no komandām ir pilnā sastāvā!

\end{problem}

%5
\begin{problem}
\textit{[Draugu daudzums]}
Pierādiet, ka jebkurā $5$ cilvēku kompānijā eksistē divi ar vienādu paziņu skaitu\footnote{šeit un turpmāk uzskatīsim, ka draudzības un pazīšanās ir abpusējas} šajā kompānijā!
\end{problem}

%6
\begin{problem}
\textit{[Unikālās atķirības]}
Doti $8$ dažādi naturāli skaitļi, kas nepārsniedz $15$. Pierādiet, ka starp šo skaitļu savstarpējām starpībām ir vismaz trīs vienādas!
\end{problem}

%7
\begin{problem}
\textit{[Pārāk daudz karaļu]}
Kādu maksimālo daudzumu karaļu var novietot uz šaha laukuma tā, lai neviens no tiem neapdraudētu citu?
\end{problem}

%8
\begin{problem}
\textit{[Pārāk daudz naudas]}
Kvadrātā $1\times 1$ metrs sameta $51$ (punktveida) monētu. Pierādiet, ka ar \mbox{$20\times 20$ cm} papīra lapu var pārklāt $3$ monētas!

\end{problem}

%9
\begin{problem}
\textit{[Kliķu neizbēgamība]}
Pierādiet, ka starp jebkuriem  $6$ cilvēkiem eksistē vai nu trīs savstarpēji pazīstami, vai arī trīs savstarpēji nepazīstami!
\end{problem}

%10
\begin{problem}
\textit{[$3^n$ pēdējie cipari]}
Pierādiet, ka eksistē skaitļa 3 pakāpe, kas beidzas ar cipariem $001$!
\end{problem}

%11
\begin{problem}
\textit{[Koledžas problēmas]}
$11$ studenti apmeklē $5$ lekcijas (ne obligāti visas). Pierādiet, ka eksistē tādi studenti $A$ un $B$, ka visas lekcijas, ko apmeklē $A$, apmeklē arī $B$!
\end{problem}

%12
\begin{problem}
\textit{[Kvadrāti no haosa]}
Kopa $A$ sastāv no $2016$ dažādiem skaitļiem, visi šo skaitļu pirmreizinātāji ir mazāki par 
\renewcommand{\labelenumi}{\alph{enumi})}
\begin{enumerate}
\item
29
\item
30
\end{enumerate}
Pierādiet, ka kopā $A$ var atrast tādus $4$ dažādus skaitļus, ka $a\cdot b \cdot c \cdot d = n^2$ kādam naturālam $n$!
\end{problem}

%13
\begin{problem}
\textit{[Bonus uzdevums 1]}
Izdomājiet Dirihlē principa vispārinājumu – $a$ hobiti saskrēja $b$ alās, un zināms, ka alas maksimālā ietilpība ir $c$...
\end{problem}

%14
\begin{problem}
\textit{[Bonus uzdevums 2]}
Vai jūs varat pierādīt Dirihlē principu?
\end{problem}

\end{document}

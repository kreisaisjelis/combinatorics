%!TEX TS-program = XeLaTeX
\documentclass[11pt]{article}

\usepackage{amssymb}
\usepackage{amsthm}
\usepackage{amsmath}
\usepackage{mathtools}

\usepackage{fancyhdr}
\usepackage{graphicx}
\usepackage[top=3cm, left=2cm, right=2cm, headheight = 90pt]{geometry}
\usepackage{xltxtra}
\usepackage[font=small,labelfont=bf]{caption}

%%%%%%%%%%%%%%    Language matters  %%%%

%\usepackage[latvian]{babel}
%\usepackage[L7x]{fontenc}
%\usepackage[utf8x]{inputenc}

%%%%%%%%%%%%%%%%%%%%%%%%%%%%%%%%%%%7%%%%%

%%%%%%%%%%%%%%%%%%%%%%%%%%%       DO NOT EDIT         %%%%%%%%%%%%%%%%%%
%\usepackage{setspace}
%\renewcommand{\headrulewidth}{1pt}
%\fancyhead[L]{\includegraphics[width=3cm]{pictures/logo}}
%\fancyhead[R]{\raisebox{3ex}{\fbox{Language: \bf \lang}}}
\fancyhead[C]{{\Large\bf Bruteforce uzdevumu risinājumi}\\ \date}
\renewcommand{\theenumi}{\alph{enumi}}

\def\leq{\leqslant}
\def\geq{\geqslant}
\def\N{\mathbb N}
\def\R{\mathbb R}
\def\Z{\mathbb Z}

\DeclarePairedDelimiter\set\{\}
\newcommand\myeq{\stackrel{\mathclap{\normalfont\mbox{def}}}{=}}

%%%%%%%%%%%%%%%%%%%%%%%%%%%%%%%%%%%%%%%%%%%%%%%%%%%%%%%%%%%%%%%%%%%%%%%%%


%%% Language name in english %%%%%%%%%
\def\lang{Latvian}

%\def\lang{Lithuanian}

%%%%%%%%%%%%%%%%%%%%%%%%%%%%% TRANSLATE HERE %%%%%%%%%%%%%%%%%%%%%%%%%%%%%%%%%%

\def\date{2018. gada 18. jūnijs}
%\def\notes{}


%%%%%%%%%%%%%%%%%%%%%%%%%%%%%%%%%%%%%%%%%%%%%%%%%%%%%%%%%%%%%%%%%%%%%%%%%%%%%%%

\def\prob{}

%%%%%%%%%%%%%%%%%%%%%%%%%%%%%%%%%%%%%%%%%%%%%%%%%%%%%%%%

\theoremstyle{definition}
\newtheorem{problem}{\prob}

\pagestyle{fancy}



\begin{document}
%\thispagestyle{fancy}
\noindent 
%\emph{\notes}


%1
\begin{problem}
$[IMO2014PL2]$
We claim the answer is $k = \lceil \sqrt{n}\rceil  - 1$, where $\lceil n\rceil$ is the ceiling function of $n$; i.e., the least integer greater than or equal to $n$. Notice that $\lceil n\rceil  < n + 1$.

First, we shall show that each $n \times n$ chessboard with a peaceful configuration of $n$ rooks contains a valid $k \times k$ square. Consider firstly the rook $R$ on the top row of the board. Because $k < n$, there exists a set $C$ of $k$ consecutive columns, one of which contains rook $R$. Consider the $n - k + 1$ $k \times k$ squares in $C$. Of them, only one contains the rook $R$ on one of its squares. Furthermore, each of the other $k - 1$ rooks in $C$ can only make up to $k$ of the $k \times k$ squares have it on one of its squares. Therefore, because \[n - k + 1 = n - \lceil \sqrt{n}\rceil  + 2 > n - \sqrt{n} + 1 = \sqrt{n} (\sqrt{n} - 1) + 1 > (\lceil \sqrt{n}\rceil  - 1) (\lceil \sqrt{n}\rceil  - 2) + 1 = k(k - 1) + 1,\] by the Pigeonhole Principle there exists a $k \times k$ square in $C$ not covered by any rook in $C$. Clearly, that square cannot be covered by any rook outside of $C$, and so it is a valid choice.

It remains to show that there exists a chessboard without a $(k+1) \times (k+1)$ square. Indeed, such a board exists. First, place a rook at the upper-left corner of the board. Next, place a rook 1 space to the right and $(k+1)$ spaces down of the first rook. Then, place a rook 1 space to the right and $(k+1)$ spaces down of the second rook, and so on, until the placement of a new rook will be under the lower boundary of the board. In that case, place a rook in the same unoccupied column and in the first unoccupied row, and continue placement of subsequent rooks. This arrangement of rooks is clearly peaceful, and because $(k+1)^2 > n$ it has the property that any $(k+1) \times (k+1)$ square in the board will be occupied by at least one rook, completing the proof.
\end{problem}


%2
\begin{problem}
$[IMO2014SLN3]$
We will show that for every positive integer $N$ any collection of Cape Town coins
of total value at most $N -\frac{1}{2}$ can be split into $N$ groups each of total value at most $1$. The
problem statement is a particular case for $N = 100$.

We start with some preparations. 

Clearly, each coin of value $1$ should form a single group; if there are $d$ such coins then
we may remove them from the collection and replace $N$ by $N - d$. So from now on we may
assume that there are no coins of value $1$.

If several given coins together have a total value also of the form $\frac{1}{k}$ for a positive integer $k$, then we may merge them into one new coin. Clearly, if the
resulting collection can be split in the required way then the initial collection can also be split.

After each such merging, the total number of coins decreases, thus at some moment we
come to a situation when no more merging is possible. At this moment, for every even $k$ there
is at most one coin of value $\frac{1}{k}$ (otherwise two such coins may be merged), and for every odd $k>1$ there are at most $k - 1$ coins of value $\frac{1}{k}$
(otherwise k such coins may also be merged).

Finally, we may split all the coins in the following way. For each $k = 1, 2, . . . , N$ we put all the coins of values $\frac{1}{2k-1}$ and $\frac{1}{2k}$
into a group $G_k$; the total value of $G_k$ does not exceed
$$
(2k-2) \cdot  \frac{1}{2k-1}+\frac{1}{2k}<1
$$
It remains to distribute the “small” coins of values which are less than $\frac{1}{2N}$; we will add them one
by one. In each step, take any remaining small coin. The total value of coins in the groups at
this moment is at most $N - \frac{1}{2}$, so there exists a group of total value at most $\frac{1}{N}(N-\frac{1}{2})=1-\frac{1}{2N}$;
thus it is possible to put our small coin into this group. Acting so, we will finally distribute all
the coins.

\end{problem}

%3
\begin{problem}
$[IMO2017SLN5]$
Split the soccer players into $N$ groups of $N+1$ consecutive people. Give the players a bandanna one by one, starting with the tallest player, then the second tallest, etc. Take the first moment when we have two players in the same group with a bandanna. Make them the first pair, and delete their group and also everyone who got a bandanna so far. We are left with $N-1$ groups of at least $N$ people. Repeat. Done.

\textit{Comment}

A number of contestants used the following incorrect approach to Problem 5: first show that there exists 2 players such that the sum of their absolute difference in heights and distance in position (both relative positions) is at most $2N$ (this could be improved to $2N-1$), and then erroneously use (weak) induction by taking these $2$ players and then ignoring everyone who is between these $2$ players in terms of either position or height.

The error here is that it is possible, in the resulting configuration (after taking the $2$ players) to inductively take two consecutive players (which would be compared by one of the conditions) on both sides of the $2$ chosen players, so the induction does not hold.

\end{problem}

%4
\begin{problem}
$[IMO2000PL4]$
Consider $1$, $2$ and $3$. If they are in different boxes, then $4$ must be in the same box as $1$, $5$ in the same box as $2$ and so on. This leads to the solution where all numbers congruent to each other mod $3$ are in the same box.

Suppose $1$ and $2$ are in box $A$ and $3$ in box $B$. Then $4$ must be in box $A$ or $B$. In general, if $k\ge 4$ is in either box $A$ or $B$, then $k + 1$ also must be in box $A$ or $B$. Thus box $C$ is empty which is impossible.

Similarly, it is impossible for $1$ and $3$ to be in box $A$ and $2$ in box $B$.

Thus we are left with the case where $1$ is in box $A$ and $2$ and $3$ in box $B$. Suppose box $B$ contains $2, . . . k$, where $k \ge 3$, but does not contain $k + 1$ and $m$ is the smallest number in box $C$. Then $m > k$.

If $m > k + 1$, then $k + 1$ must be box $A$ and we see that no box can contain $m-1$.

Thus $m = k + 1$. If $k < 99$, we see that no box can contain $k + 2$. Thus $k = 99$. It is easy to see that this distribution works. Thus there altogether $12$ ways - $2$ options times permutation of $3$ colors for each of $3$ boxes.
\end{problem}
%5
\begin{problem}
$[BW2002]$
Let the three chosen numbers be $(x_1, x_2, x_3)$. At least one of the sets $\set{1,2,...,24}$, $\set{25,26,...,48}$, $\set{49,50,...,72}$ and $\set{73,74,...,96}$ should contain none of $x_1, x_2$ and $x_3$; let $S$ be this set.

We split $S$ into six parts $S=S_1 \cup S_2 \cup ... \cup S_6$ so that the first four elements of $S$ are in $S_1$, the next four in $S_2$, etc. 

Now we choose $i \in \set{1,2,...,6}$ corresponding to the order of numbers $x_1, x_2$ and $x_3$ (if $x_1 < x_2 < x_3$ then $i=1$, if $x_1 < x_3 < x_2$ then  $i=2$, ..., if $x_3 < x_2 < x_1$ then  $i=6$).

Lastly, let $j$ be the number of elements in $\set{x_1, x_2, x_3}$ that are greater than the elements of $S$ (note that any any $x_k$ is either greater or smaller than all elements of $S$)

Now we choose $x_4 \in S_i$ so that $x_1+x_2+x_3+x_4 \equiv j$ mod $4$ and add the card number $x_4$ to those three cards.

Decoding of $\set{a,b,c,d}$ is straightforward. We first put the numbers into increasing order and then calculate $a+b+c+d$ mod $4$ showing the added card. The added card belongs to some $S_i$ for some $S$ and $i$ shows us the initial ordering of cards.

\end{problem}
%6
\begin{problem}
$[Bonus]$
tbd
\end{problem}


\end{document}

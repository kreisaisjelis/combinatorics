%!TEX TS-program = XeLaTeX
%!TEX TS-program = XeLaTeX
\documentclass[11pt]{article}

\usepackage{amssymb}
\usepackage{amsthm}
\usepackage{amsmath}
\usepackage{mathtools}

\usepackage{fancyhdr}
\usepackage{graphicx}
\usepackage[top=3cm, left=2cm, right=2cm, headheight = 90pt]{geometry}
\usepackage{xltxtra}
\usepackage[font=small,labelfont=bf]{caption}

\usepackage{multicol}

\renewcommand{\theenumi}{\alph{enumi}}


\def\leq{\leqslant}
\def\geq{\geqslant}
\def\N{\mathbb N}
\def\R{\mathbb R}
\def\Z{\mathbb Z}
\DeclarePairedDelimiter\set\{\}

\def\prob{}

\theoremstyle{definition}
\newtheorem{problem}{\prob}


\pagestyle{fancy}

%!TEX TS-program = XeLaTeX

\fancyfoot[CE,CO]{}  % this is to remove page numbers (as you might want for single page docs)

%!TEX TS-program = XeLaTeX
\renewcommand{\figurename}{Attēls}

\fancyhead[C]{{\Large\bf Invarianti - Uzdevumi}\\ \date}


\begin{document}

\noindent 
%\emph{\notes}



%1
\begin{problem}
\textit{[Spēle uz tāfeles]}
Uz tāfeles sarakstīti $2017$ skitļi $1$ un $2018$ skaitļi $-1$. Vienā gājienā atļauts izvēlēties divus skaitļus, kas uzrakstīti uz tāfeles, tos nodzēst un tā vietā uzrakstīt vienu skaitli $1$, ja nodzēstie skaitļi bija vienādi un $-1$, ja tie bija dažādi.

Kāds būs pēdējais skaitlis, kas būs uzrakstīts uz tāfeles, veicot šādus gājienus?

\end{problem}
%

%2
\begin{problem}
\textit{[Monētu spēle]}
Luīzei ir viena monēta. Viņa ir uzgājusi naudas maiņas mašīnu, kura, ievietojot tajā jebkādu monētu, atgriež $5$ monētas.
Vai ar šīs mašīnas palīdzību Luīze var iegūt tieši $2018$ monētas?
\end{problem}
%

%3
\begin{problem}
\textit{[Zirdziņu šahs]}
Ja zirdziņš sāk uz lauciņa $a1$, vai ir iespējams apciemot katru lauciņu tieši vienu reizi un pabeigt ceļojumu uz lauciņa $h8$?
\end{problem}
%

%4
\begin{problem}
\textit{[Akmentiņu spēle]}
Spēles sākumā uz galda ir viena kaudzīte ar $2018$ akmentiņiem. Vienā gājienā ir atļauts paņemt vienu kaudzīti, vienu akmentiņu no tās nomest uz grīdas, bet atlikušos sadalīt divās (ne obligāti vienādās) kaudzītēs, kuras nolikt atpakaļ uz galda. 

Vai kādā šīs spēles brīdī ir iespējama situācija, ka katrā akmentiņu kaudzītē, kas atrodas uz galda, ir tieši $3$ akmentiņi?
\end{problem}
%

%3
\begin{problem}
\textit{[Apļa spēle]}
Seši skaitļi - $(0,1,0,1,0,0)$ - sarakstīti pa apli. Vienā gājienā atļauts pieskaitīt $1$ diviem uz šī apļa blakus stāvošiem skaitļiem.

Vai ir iespējams sasniegt situāciju, kurā visi skaitļi uz šī apļa ir vienādi?
\end{problem}
%

%n
\begin{problem}
\textit{[Picas spēle]}
Apaļa pica sastāv no  $n\ge3$ šķēlēm. Sākumā viena no šķēlēm ir apmesta ar sieru uz leju. Vienā gājienā atļauts izvēlēties vienu šķēli, kas ir ar sieru uz leju, apmest to ar sieru uz augšu un divas blakusstāvošās šķēles apmest otrādi.

Kādām $n$ vērtībām iespējams sasniegt stāvokli, kad visas picas šķēles ir ar sieru uz augšu?
\end{problem}
%

\end{document}

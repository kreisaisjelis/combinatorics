%!TEX TS-program = XeLaTeX
%!TEX TS-program = XeLaTeX
\documentclass[11pt]{article}

\usepackage{amssymb}
\usepackage{amsthm}
\usepackage{amsmath}
\usepackage{mathtools}

\usepackage{fancyhdr}
\usepackage{graphicx}
\usepackage[top=3cm, left=2cm, right=2cm, headheight = 90pt]{geometry}
\usepackage{xltxtra}
\usepackage[font=small,labelfont=bf]{caption}

\renewcommand{\theenumi}{\alph{enumi}}

\fancyfoot[CE,CO]{}  % this is to remove page numbers (as you might want for single page docs)

\def\leq{\leqslant}
\def\geq{\geqslant}
\def\N{\mathbb N}
\def\R{\mathbb R}
\def\Z{\mathbb Z}
\DeclarePairedDelimiter\set\{\}

\def\prob{}

\theoremstyle{definition}
\newtheorem{problem}{\prob}


\pagestyle{fancy}

%!TEX TS-program = XeLaTeX

\fancyfoot[CE,CO]{}  % this is to remove page numbers (as you might want for single page docs)

%%!TEX TS-program = XeLaTeX
\renewcommand{\figurename}{Attēls}

\fancyhead[C]{{\Large\bf Invariant - Problems}\\ \date}


\begin{document}

\noindent 
%\emph{\notes}



%1
\begin{problem}
\textit{[Blackboard game]}
There are $2017$ $1$'s and $2018$ $-1$'s written on a blackboard. A move allows to choose and erase two numbers from a blackboard and write $1$ if erased numbers were equal, and $-1$ if they were different. 

What number will be the last on the blackboard?
\end{problem}
%

%2
\begin{problem}
\textit{[Coin game]}
Luize has one coin, and she comes upon a coin-exchange machine, that exchanges one coin for $5$ other coins. 

Can she, by using this machine, get exactly $2018$ coins? 
\end{problem}
%

%3
\begin{problem}
\textit{[Knight chess game]}
If a knight starts at $a1$ is it possible to visit all the chess field squares exactly once and finish on $h8$?
\end{problem}
%

%4
\begin{problem}
\textit{[Stones game]}
Initially there is one pile of $2018$ stones on the table. In a move, it is allowed to take one pile of stones, remove one stone from it and then split the pile into two (not necessarily equal) piles. 
Is it possible that at some point in this game all the piles on the table will have exactly $3$ stones in each of them?
\end{problem}
%

%5
\begin{problem}
\textit{[Circle game]}
Six numbers - $(0,1,0,1,0,0)$ - are written in a circle. A move allows adding $1$ to each of two neighbouring numbers. 

Is it possible to reach a situation where all numbers in a circle are equal?
\end{problem}
%

%6
\begin{problem}
\textit{[Pizza flipping - PAMO2018PL6]}
A circle consists of $n\ge3$ numbers, initially, one number is $0$ all others are $1$. A \textit{move} means choosing a number $0$ changing it to $1$ and then changing two neighboring numbers to the opposites $0\Leftrightarrow1$.

For what values of $n$ is it possible to reach a configuration where all numbers are $1$?
\end{problem}
%

%7
\begin{problem}
\textit{[Golden cards - IMO2007 SLC1]}
Consider $2009$ cards, each having one gold side and one black side, lying in parallel on a long table. Initially all cards show their gold sides. Two players, standing by the same long side of the table, play a game with alternating moves. Each move consists of choosing a block of $50$ consecutive cards, the leftmost of which is showing gold, and turning them all over, so those which showed gold now show black and vice versa. The last player who can make a legal move wins.
\begin{enumerate}
\item Does the game necessarily end?
\item Does there exist a winning strategy for the starting player?
\end{enumerate}
\end{problem}
%

%8
\begin{problem}
\textit{[Number pair game - IMO2012 SLC1]}
Several positive integers are written in a row. Iteratively, Alice chooses two adjacent numbers $x$ and $y$ such that $x > y$ and $x$ is to the left of $y$, and replaces the pair $(x, y)$ by either $(y + 1, x)$ or $(x − 1, x)$. Prove that she can perform only finitely many such iterations.
\end{problem}
%


\end{document}

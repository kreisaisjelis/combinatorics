%!TEX TS-program = XeLaTeX
\documentclass[11pt]{article}

\usepackage{amssymb}
\usepackage{amsthm}
\usepackage{amsmath}
\usepackage{mathtools}

\usepackage{fancyhdr}
\usepackage{graphicx}
\usepackage[top=3cm, left=2cm, right=2cm, headheight = 90pt]{geometry}
\usepackage{xltxtra}
\usepackage[font=small,labelfont=bf]{caption}

%%%%%%%%%%%%%%    Language matters  %%%%

%\usepackage[latvian]{babel}
%\usepackage[L7x]{fontenc}
%\usepackage[utf8x]{inputenc}

%%%%%%%%%%%%%%%%%%%%%%%%%%%%%%%%%%%7%%%%%

%%%%%%%%%%%%%%%%%%%%%%%%%%%       DO NOT EDIT         %%%%%%%%%%%%%%%%%%
%\usepackage{space}
%\renewcommand{\headrulewidth}{1pt}
%\fancyhead[L]{\includegraphics[width=3cm]{pictures/logo}}
%\fancyhead[R]{\raisebox{3ex}{\fbox{Language: \bf \lang}}}
\fancyhead[C]{{\Large\bf Invariant - Problems}\\ \date}

\renewcommand{\theenumi}{\alph{enumi}}
%\newcommand{\problem}[1]{\paragraph{Problem #1.}}%<--------------- TRANSLATE THE WORD "Problem".
\fancyfoot[CE,CO]{}  % this is to remove page numbers (as you might want for single page docs)

\def\leq{\leqslant}
\def\geq{\geqslant}
\def\N{\mathbb N}
\def\R{\mathbb R}
\def\Z{\mathbb Z}
\DeclarePairedDelimiter\set\{\}
\newcommand{\?}{\stackrel{?}{=}}

%%%%%%%%%%%%%%%%%%%%%%%%%%%%%%%%%%%%%%%%%%%%%%%%%%%%%%%%%%%%%%%%%%%%%%%%%


%%% Language name in english %%%%%%%%%
\def\lang{Latvian}

%\def\lang{Lithuanian}

%%%%%%%%%%%%%%%%%%%%%%%%%%%%% TRANSLATE HERE %%%%%%%%%%%%%%%%%%%%%%%%%%%%%%%%%%

%\def\date{2018. gada 18. jūnijs}
%\def\notes{}


%%%%%%%%%%%%%%%%%%%%%%%%%%%%%%%%%%%%%%%%%%%%%%%%%%%%%%%%%%%%%%%%%%%%%%%%%%%%%%%

\def\prob{}

%%%%%%%%%%%%%%%%%%%%%%%%%%%%%%%%%%%%%%%%%%%%%%%%%%%%%%%%

\theoremstyle{definition}
\newtheorem{problem}{\prob}

\pagestyle{fancy}



\begin{document}
%\thispagestyle{fancy}
\noindent 
%\emph{\notes}


%1
\begin{problem}
\textit{[Blackboard game]}
There are $2017$ $1$'s and $2018$ $-1$'s written on a blackboard. A move allows to choose and erase two numbers from a blackboard and write $1$ if erased numbers were equal, and $-1$ if they were different. 

What number will be the last on the blackboard?
\end{problem}
%

%2
\begin{problem}
\textit{[Coin game]}
Luize has one coin, and she comes upon a coin-exchange machine, that exchanges one coin for $5$ other coins. 

Can she, by using this machine, get exactly $2018$ coins? 
\end{problem}
%

%3
\begin{problem}
\textit{[Knight chess game]}
If a knight starts at $a1$ is it possible to visit all the chess field squares exactly once and finish on $h8$?
\end{problem}
%

%4
\begin{problem}
\textit{[Stones game]}
Initially there is one pile of $2018$ stones on the table. In a move, it is allowed to take one pile of stones, remove one stone from it and then split the pile into two (not necessarily equal) piles. 
Is it possible that at some point in this game all the piles on the table will have exactly $3$ stones in each of them?
\end{problem}
%

%3
\begin{problem}
\textit{[Circle game]}
Six numbers - $(0,1,0,1,0,0)$ - are written in a circle. A move allows adding $1$ to each of two neighbouring numbers. 

Is it possible to reach a situation where all numbers in a circle are equal?
\end{problem}
%

%n
\begin{problem}
\textit{[PAMO2018PL6]}
A circle consists of $n\ge3$ numbers, initially, one number is $0$ all others are $1$. A \textit{move} means choosing a number $0$ changing it to $1$ and then changing two neighboring numbers to the opposites $0\Leftrightarrow1$.

For what values of $n$ is it possible to reach a configuration where all numbers are $1$?
\end{problem}
%

\end{document}
